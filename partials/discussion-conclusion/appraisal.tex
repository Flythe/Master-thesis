\section{Appraisal}

\silvia{im not sure to understand the term "appraisal", other than evaluation and discussion that was already presented in other sections. what is the difference here? is this following some other line of thought? }

\paragraph{Strengths and limitations}
This section will be split into two parts, as there are strengths and limitations considering the system but also the implementation itself.

Starting with the implementation, the biggest limitation of the study is the limited time allowed for development due to struggles with data gathering.
This resulted in many design shortcomings in the front-end and minimal implementation of security features.

Features that are not directly visible in the front-end are not critical for a successful demo.
However, a \silvia{?clean} user experience with the system should be reached before pitching all the systems' benefits.
This will increase trust, where `invisible' back-end related functions can be expanded upon orally during a demo.

\silvia{i feel this has been said before:}
Even though the thesis contains a rather extensive research into security, most of the found solutions were not used in the \ivfsystem{}implementation yet.
Even now, after the first prototype is finishe, it would not be advisable to start introducing these to the system.
First another iteration of function and design development should be ran before it becomes feasible to use intricate security constructs.
For demo purposes this is not a big problem as long as security issues and solutions are kept in mind before starting to use the system in a live environment.

While the system's prototype might be lacking \silvia{..what?}, it already illustrates the potential after some more development effort.
In the evaluation users noted that the ideas of research workflow management can attract investors. \silvia{really?}
Value of an investment into data gathering is increased when sharing and reuse capabilities are added.
As found in \silvia{?the big data story},  these two things are vastly important in current day grant requests.

\silvia{i dont understand this : Next to these inherit profits of a system like this,} it also supports direct users (\ie{} researchers, data manager, committee members).
Researchers are currently supported along a big part of their research workflow.
While the data manager and committee members have an all-in-one system for communication and management tasks.
As said before, with another development cycle this side of the system's potential can be fully shown.

When considering the system itself, and how it is currently placed in one specific domain, it is hard to tell how it will extrapolate to other domains.
Because the data model of the implementation is very flexible this should not lead to problems.
However, the set-up is currently that the dataset is fixed after system initialisation.
This implicates at least two things: there is no data input or edits, and consent for data use is handled by the research committee only.
Data editing and input are functional requirements which are easily implemented.
Consent and lawful procedures are quite  challenging and might change between domains.

One last limitation of the system is that for reuse only open-source software was considered.
This is also due to time restrictions.
However, the drawbacks here are that well implemented existing systems with functions like data gathering or data export were not considered.
Using these as building blocks in a larger cooperating framework might be more advantageous.
This would also mean that a gateway has to be developed to encompass all the different systems into one interface, otherwise this might \silvia{?rebuff} prospective users.


\silvia{after reading this complete section i still have the feeling that a lot is repetition of what has been said before, or that would fit in 5.2 anyways}

\paragraph{Comparison with existing systems}
Multiple systems exist which perform data management functions, for example OpenClinica and DADOS prospective amongst others.
These systems were considered in the software reuse step because they are open-source.
Of course there are many more systems that perform about the same functions, where each of them will have a slightly different implementation.
For the sake of this section all these systems will be considered as Clinical Trial Management (CTM) software.

Each of the reviewed CTM systems at least provide data gathering and data retrieval functionality.
Most of them implement gathering with flexible input form creation which, after defined, are to be strictly followed during a certain trial.
Data retrieval (mostly) consists of a downloadable formatted file of the data, specific for statistical packages like SPSS.

For the goals we are trying to reach with d-gate, these functions only fulfil a small part of the requirements. \silvia{really? do you have data entry requirements for d-gate?}
This especially became clear after the brainstorm which added many more requirements outside of the data management part.
Therefore the CTM systems may be used as a building block in a system like the \ivfsystem{}, but it will never completely fulfil the user's needs for  management of data reuse.
Reuse of CTM software is advisable when the data model has to be well defined and supported (\ie{} many researchers use a specific CTM which proves that the data model is correct).
\silvia{but this could be added to the system as a data import function, which is currently totally offline}

\paragraph{Future research and development considerations}
\silvia{how does this paragraph relate to the previous "further development"?}
The \ivfsystem{} should be further developed and used in further data management research in the clinical domain.
In this domain a vast network of interweaving interests, guidelines, and laws exists.
Untangling this network should be the aim of further research into this area.
Research could be done on different levels: government and lawmakers, hospitals and clinics, or patient groups.
It also could approach different perspectives: ethical, technical, or social.
Furthermore, research into further development of the \ivfsystem{} can be performed.

\silvia{these are interesting questions, but i do not see their value without a clear origin or more discussion about how to perform this. It now reads as a random wish list of things with totally different nature and scale, and without much substance. }
Examples of research questions are:

\begin{itemize}
	\item How can clinician be persuaded to provide patient data for research?
	\item What type of technology can be used to improve data reuse considering patient consent?
	\item How should data be structured to maximise research benefits while making sure privacy is kept?
	\item How can user expectations for the \ivfsystem{} be managed?
	\item How can the system provide better support to users while they perform data management tasks?
\end{itemize}