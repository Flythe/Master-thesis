\section{Conclusion}

The focus of the developed system changed during this study.
Initially the system would be supporting researchers in doing research which was the basis for the first research question:
How do we implement a user-friendly system in a \IVF{}--\PRN{} medical domain which covers problems concerning: data security, data access, data browsing, and data querying?
The conclusion to this question will be structured through its sub-questions.

While the requirements have changed, there are still aspects of the system which attribute to the problems we wanted to address.
Required functions which should be supported and the prospective users have been identified.
These are listed in section \ref{brainstorm} in figure \ref{fig:brainstorm-after} and section \ref{functional-design} in figure \ref{fig:functions-workflow}.
Researchers, the data manager, and committee members need to be supported in functions belonging to user, data, request, and publication management.
Most notable is that the focus of the system now lies on request management and the committee members.

Of course, because the \projectdata{} contains sensitive medical data, security is an important aspect of the system.
An extensive security review, both in literature as with an interview, lists the possible issues and proposes solutions to them.
This review can be found in appendix \ref{security-appendix} and a summary is described in section \ref{security}.
Security solutions are mostly used during the fertility data gathering process, removing most of the privacy sensitive data.
No solutions were implemented in the system due to time restrictions.

To store the data in the system the data model of an existing system was reused.
This is the in-house project Rosemary, originally   developed for the neuroscience domain.
With minimal changes the data model could fit to the \projectdata{}, which is described in section \ref{implementation} and shown in figure \ref{fig:implementation-rosemary-dm}.

Functions which would be implemented into the \ivfprototype{} were decided upon with educated guesses.
The basic functions like data access, browsing, and querying were mostly covered through the reuse of Rosemary.
Most extensive development went into request management and supporting the committee users.
This prototype is not completed yet.
More functionality should be added, but most important the user interface should be redesigned.
This could, for example, be done with a user-centred design process.
However, the prototype shows what a completed system is capable of which is reflected in the evaluation in chapter \ref{evaluation}.

The second research question considers the underlying processes of a system like the \ivfsystem{}.
It tries to tie together the \projectdata{} with the concepts of big data with the research question: What needs to be changed in the current attitude towards data usage to promote big data in a \IVF{}--\PRN{} medical domain?
The conclusion to this question will be structured through its sub-questions.

Big data literature is mostly about possibilities and positive aspects.
Aspects related to the system's dataset are described in `What is big data?' and `Big data for \project{} research' in chapter \ref{introduction}.
Most notable are preservation, reuse, and analysis or decision making.
The literature describes that funding institutions push for effort into data preservation.
These efforts should lead to more data reuse even after the end of the funding grant.

The strength of big data lies with the possibilities of decision making.
Computers are better capable to analyse data and find correlations within them.
In the case of the \project{} this could be used to determine what possible hypothesis can be answered with the data.
This idea of data analysis was captured in the system's initial concept (section \ref{process-analysis} figure \ref{fig:brainstorm-before}).
During the brainstorm session these functions were considered `nice to have', but not essential.
Hypothesis are formulated by researchers and then tested on the \projectdata{}.
To support this,  preservation and reuse aspects are important.
Data should be accessible for researchers that want to test a certain hypothesis.
However, the sensitive nature of the data and the associated regulations make this difficult.
This is shown in the consent discussion in the security review (section \ref{security}).
But also in the fact that committee members have become  important users of the system.

Preservation and reuse are addressed in the \project{} project with the support of \ivfsystem{}.
We discussed that through transparency and giving back control over data the owners are more willing to expose their data for research ends.
Analysis is not currently in the system and therefore this question remains open.
We expect that the system in the future can overcome some (if not all) of the identified blocking aspects.
However, further research where the system is used in production should show this.