\section{Implications}

\subsection{Overview}

In the Netherlands about 5\% to 8\% of all couples remain childless due to infertility or subfertility.
There are several treatments used to assist in reproduction these are among others: IUI, IVF, and ICSI.
While it is known how many treatments result in a pregnancy it is relatively unknown what the outcomes are for children born out of these pregnancies.
For this end the \project{} was started to gather and analyse data from both the fertility clinics as the national birth registry (\ie{} \PRN{}).

Because considerable effort went into data gathering reuse of data should be supported to get the most gain out of it.
From this the idea for the \ivfsystem{} was formulated, a system supporting data management.
During the requirement analysis it appeared that there are many more aspects which lead to effective research (and thus, data reuse).
Identified requirements were separated into management groups: user, request, data, and publication.

Data management still remained the most extensive requirement group, thus for software reuse three systems were evaluated for this.
An in-house developed project was chosen as a starting point for back-end and front-end.
With minimal changes to the data model it was possible to store and use the \project{} data.
Due to time restrictions user, request, and data management functions were implemented partly.
Publication management was left unimplemented all together.

During the evaluations it became clear that for a successful demo more `fancy bits' had to be added.
Also one or two iterations of user-centred design should be applied to remove most of the fundamental confusions in the interface.

\subsection{Connecting the dots}

%How do we implement a user-friendly system in a IVF medical domain which covers problems concerning: data security, data access, data browsing, and data querying?
%What needs to be changed in the current attitude towards data usage to promote big data in a IVF medical domain?

\paragraph{Supporting administrative tasks}

During the development of the \ivfsystem{} many hurdles in the current research workflow came to light.
Normally these might be hidden to outside spectators as researchers have found workarounds.
This paragraph will discuss what the importance of supporting administrative tasks is.

Starting with one of the most important aspects for sensitive data: security.
It is known that when humans are involved errors are inevitable when time goes by.
Humans make errors, but it is possible to help them in making less errors.
Providing overviews and insights in the process they are working with can achieve prevention.

As shown in the development of the \ivfsystem{} multiple actors are involved in the data management process.
A lot of communication will run between these actors, currently this is mostly done asynchronously and sometimes even offline.
This problem gives an clear example of security through structure.
Providing the actors with a communication channel keeps the conversation encapsulated in a single system, and therefore in a single dataset.
This dataset can be leveraged to provide managers with insight in the process and can help them spot possible errors and prevent these.

Lastly, there is also inherent technical security from a well designed system.
If all actors are involved in the development and execution of the system this will lead to a better technical security as well.
In a properly executed development cycle all requirements should be identified and then analysed on whether they are important for the system.
Because specific domain knowledge is necessary when creating a system in a clinical domain, in order to identify all requirements many experts should be involved.
Each of these will input their ideas giving a better coverage on the flaws that should be expected.

A well developed (and documented!) system also leads to another aspect: understanding of underlying processes.
The system should be the embodiment of existing research processes, sometimes parts of these can be `black boxed'.
Meaning that for people outside of the `box' it is unknown what is going on and what step the process is at.
Supporting the black box processes gives the possibility to provide outsiders with cherry picked or aggregated data.

This idea can prove useful for processes like: security, restriction, approval.
Right now these are mostly hidden from data consumers but also from data deliverers.
Involving these actors will lead to understanding and ultimately to trust.
Trust can be directed towards the system but also the institution providing the system, in this study this is also the data gatherer and owner.

In that sense the system can be used as a marketing product.
Laying bare the inner workings of the institution and showing that errors might exist but also how they are resolved.
Next to data providers institutions providing grants are highly important.
Nowadays they require that their investment in data gathering should lead to high returns for the scientific community.
Mostly this will mean that data reuse is of high priority. 
Supporting a grant request with the \ivfsystem{} will show exactly how the requester plans to provide this.

Ultimately the goal of the \ivfsystem{} is to provide reuse capabilities for datasets.
Unfortunately in research (especially in the clinical domain) this is not something that comes naturally.
Data has value, a lot of effort goes into gathering and upkeep.
Researchers and institutions want to hold on to their data or at least gain something from reuse.

When making data open access it is hard to check whether if the right references are used in publications.
On the other hand data reuse can lead to better research, for example publications can be checked by an independent institute.
With a \ivfsystem{} data can be restricted while the line of communication between data owner and consumer is short.
This will bring all the benefits of data reuse while the data owner still has a grip on what happens with it.

\paragraph{Extrapolate to other domains}
%How can this specific system be used in other settings/knowledge domains?

The previous section argued the benefits of a \ivfsystem{}, however it is specific to the domain of \IVF{} and \PRN{} data.
There are other projects building towards something similar, but no complete system exists yet.
Most follow a stepwise implementation of blocks, introducing user, data, request, and publication management one by one.
Leaving each domain to their own experts however will create a fragmentation of many different implementations of (in principal) the same functionality.

It is a good idea to work towards a modular solution which implements core elements of each of the management groups.
Different domains can pick what they need and add their specific `sugar' while the functionality remains largely the same.
This will require an investment into research on how to make the system's modules abstract enough for reuse in other domains.
And even when abstract modules exist process and requirement analysis for each implementation project will take a considerable amount of time.

Before the \ivfsystem{} is suitable for this type of reuse many hurdles are yet to be taken.
These lie both in social fields as in regulation.
The social aspects relate to building trust, managing expectancy, etc.
Regulations in the Netherlands for sensitive data are strict (as to be expected).
The \IVF{} and \PRN{} dataset is relatively light on sensitive data and skirts around some of the laws.
When more interest is shown in the \ivfsystem{} these are some of the interesting areas for further research.

Furthermore, the \ivfsystem{} is not ready for demoing purposes yet.
The evaluation of the prototype shows there are areas that need improvement.
Some are time consuming and are not directly visible, like integrating data provenance.
Others will take less time and are highly visible, like updating the user interface.
Once more, this study answers the question of what is needed to implement a successful system but does not deliver it.

\paragraph{Blocking factors}
\allard{TODO once position paper is done}
%What are the biggest blocking factors in doing research with sensitive data and how can these be overcome?



