
%TODO: Check if terminology is explained, IVF, PRN, clinical audit registration, \ivfsystem{}, \project.

\silnote{add ToC}


Right now success percentages of IVF clinics are being published as this is required by law, however they complain that this is unfair as the patient mix between clinics is ‘unfair’. 
IVF clinics want to cooperate in the research of the AMC because they think it is important but they are scared that outcomes will be published in a way that will reflect directly on individual clinics.
In the longer run the AMC wants to profile itself as a knowledge hub for the clinics. The clinics can come to the AMC with questions about the data that is in the project and the AMC can provide answers, advice, or a (sub)set of the data.
As ultimate goal to give the project value the plan is to provide a quality assurance service for the government and the clinics themselves.
This idea can be explored with the NICE registration, they do something like this for Intensive Care units in the Netherlands. 
A doctor can find his or her hospital in the database and compare it versus a subset of ten comparable hospitals or versus the all the hospitals in the Netherlands. 
This registration is used to find points where improvement can be made or as a quality assurance.
Because the clinics are scared that the data will be used against them it will be important to show value of the system and make data collection and security transparent.
Most of the clinics are using the same database software called LSFD. 
To extract data it is no problem to create a query on the system but knowledge of the system is needed before we can do this, thats why we need an appointment with someone who has this knowledge. 
I will try to make an appointment together with Martina (PhD) to visit someone involved in the development of this software to find out how we can extract data from it. 
At this appointment I will try to find out if a direct extraction from a automated system without involvement of personnel is possible.
Anita found a problem in the clinical questions, mine and Alexanders were basically the same. 
I’ve changed my question and will leave it open for change during the project as Martina will also pick up one of the questions. 
The possible subjects are: IVF ICSI, fresh/cryo, type of medium, oxygen tension (?, zuurstofspanning in Dutch).

The proposed problem.
The real problem (after brainstorm).

\silnote{put here an overview}
