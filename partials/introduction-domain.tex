\paragraph{The domain and background}
\allard{change this first block}
Reproduction is a fundamental building block of life.
For the human species this means that two individuals, with a different sex each, produce offspring.
The offspring contains the genetic material of both the parents.

However, there are many conditions and diseases which can lead to infertility or subfertility.
In the Netherlands these terms are defined in a national guideline by the Dutch association of obstetrics and gynaecology (NVOG)\footnote{Dutch: Nederlandse vereniging voor obsetrie en gynaecologie}\cite{subfertilityGuideline}.
Infertility being a rare condition where ``no chance of reproduction exists''.
Where subfertility is defined as ``failure to become pregnant after twelve months of unprotected coitus aimed at conception''.

Luckily there are several fertility treatments.
Some of these lead to both the parents being biological parents. 
Others make use of donor material or surrogates, meaning that the child does not contain the genetic material of one of the `parents'.
Commonly used treatments include intrauterine insemination (IUI) and in vitro fertilisation (IVF) \cite{treatmentExplanation}.
Of these IUI is closer to regular impregnation, sperm cells are collected and injected deep in the uterus, while IVF happens `in vitro' (\ie{} in glass, outside of the body).

Often in research there is a fresh and frozen group, material of the male or female is then kept in (frozen) storage before being used.
IVF can be further divided into more specific treatment types (\eg{} intracytoplasmic sperm injection, ICSI).
The difference between these being the used technique or the type of paternal materials used.

Each treatment follows about the same steps: egg maturation stimulation, egg retrieval, fertilisation, and embryo transfer \cite{treatmentExplanation}.
The stimulation phase can also be called the start of a new \emph{cycle}, in the Netherlands (according to the NVOG) 14,562 of these cycli were started in 2013 \cite{ivfReportNVOG2013}.
Approximately 30\% of these cycli resulted in a ongoing pregnancy.
It is fairly well known what the success rate is for a given fertility clinic.
However, outcome quality indicators related to the (born) child are either sparse or unknown.

All births in the Netherlands have to be entered into the perinatal registry (perinatale registratie Nederland, PRN\footnote{http://www.perinatreg.nl}).
For research ends however this data is completely separated from the clinic's patient data.
The Dutch healthcare system is quite exceptional as fertility clinics are in the public domain.
Meaning that there is pressure for disclosing patient data for research (and governance) ends.

With minimal identifying data from both the fertility clinics and the PRN, treatment input and outcome can be linked together.
To execute this the \project{} was established.
During the project, data from each of the thirteen Dutch fertility clinics from between 1999 and 2010 is gathered and linked.
This data covers ongoing pregnancies as a child has to be born in order to link to the PRN.
In the given date range about 44,164 ongoing pregnancies were registered in the clinic data \cite{ivfReportNVOG}.
Linkage will inevitably result in a loss of a few percent where no appropriate match can be found.
But a considerate amount of pregnancies should be available for research.