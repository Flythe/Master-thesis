% Styling commands for the used terms
\newcommand{\agent}{{\tt agent}}
\newcommand{\entity}{{\tt entity}}
\newcommand{\activity}{{\tt activity}}
\newcommand{\relation}{{\tt relation}}
\newcommand{\relations}{{\tt relations}}
\newcommand{\attributes}{{\tt attributes}}

\section{Data Provenance}
\label{datamodel-provenance}

A topic with growing interest in the \escience{} field is provenance, sometimes also referred to as lineage or pedigree.
It has been borrowed from the world of art where it describes the `life' of an artwork.
Mostly this will be the record of ownership but it can also describe things like restorations.
From provenance data the quality, state, and originality of the work can be discerned.

In \escience{} the same can be applied on a piece of data \cite{dsp4moreau}.
Concerning data, provenance is stored metadata describing the process by which the data got to a certain state from a specific source \cite{dsp4moreau,dsp2buneman}.
To describe the path data has taken W3C has made a standard described in the PROV Model Primer \cite{dsp8gil}.
The gist of provenance is that it is build from a small set of assertions made by the different services that are involved in  the data process \cite{dsp4moreau}.

To actually \emph{use} provenance data a standard for schemas is described which are usable for human consumption, an example is shown in figure \ref{fig:provenance-large-schema}.
Because these schemas can run from the starting point where provenance data was kept (or actually the beginning of all time) user-tailored queries should be applied \cite{dsp4moreau}.
These frame the specific question a user has and only display the applicable part of the schema.

\paragraph{The `why?' of provenance}
\label{provenance-why}

When collecting provenance various metadata has to be captured at different steps in the data process.
This creates a overhead when using a system for a specific task.
Capturing and keeping provenance data is almost never the main functionality of a system.
However, exposing the data may help users (\ie{} researchers).

In \escience{} data is gathered and generated at a fast pace.
Provenance kept of this data can help researchers determine whether data is:

\begin{itemize}
	\item Usable in a certain context, the metadata stored can describe the different uses of a specific data item \cite{dsp1simmhan}, \eg{} types of software that accept a data item as input.
	\item Acceptable, the path a piece of data has taken to get to its current form can tell a researcher if they trust the accuracy and timeliness and accept if for further use \cite{dsp1simmhan,dsp3buneman}.
	\item Protected by intellectual property (IP) or should be credited, as for the acceptability of data the path can also be backtracked to the original creators and/or IP holders \cite{dsp1simmhan}.
\end{itemize}

\paragraph{The `how?' of provenance}
\label{provenance-how}

The provenance building blocks (as described by the W3C \cite{dsp8gil}) are the three core data types (\agent{}, \entity{}, and \activity{}) and a few of the possible \relations{} between them as shown in figure \ref{fig:provenance-overview}.
Furthermore, \attributes{} (not shown) can be assigned to provide more metadata for data types or \relations{}.
Provenance output is also standardised, figure \ref{fig:provenance-overview} shows the display methods for the data types and \relations{}, \attributes{} are displayed with a `document' symbol as shown in figure \ref{fig:provenance-large-schema}.

\begin{figure}[h]
	\centering
	\includegraphics[width=0.5\linewidth]{images/provenance-overview}
	\caption{Example model showing the three core data types (\agent{}, \entity{}, and \activity{}) and a few of all possible \relations{} between them. 
		Taken from PROV Model Primer \cite{dsp8gil}.}
	\label{fig:provenance-overview}
\end{figure}

List of used concepts as described in PROV Model Primer \cite{dsp8gil}:

\begin{itemize}
	\item Entity, physical, digital, conceptual, or another type of `thing'.
	\item Activity, the process of instantiating or the process of changing an entity.
	\item Agent, holds (a part of) the responsibility for activities and entities.
	\item Relation, describes the interaction between two instances of the three core data types.
\end{itemize}

\begin{figure}[!t]
	\centering
	\includegraphics[width=1.0\linewidth]{images/provenance-large-schema}
	\caption{
		Real life example model which implements the model as shown in figure \ref{fig:provenance-overview}.
		This example describes the creation of a chart, the original data used, the intermediate data generated during the process, the used software, who was responsible for the work, and who this person was working for.
		An addition to figure \ref{fig:provenance-overview} is the use of \attributes{}, these are displayed with document icons and provide metadata on the object they are bound to.
		In this case that is the name and email address for one the agents and the company name for the other agent.
		Taken from PROV Model Primer \cite{dsp8gil}.
		}
	\label{fig:provenance-large-schema}
\end{figure}

An provenance example is given in figure \ref{fig:provenance-large-schema}.
Here is shown what the provenance of a certain `chart' is.
The chart (\entity) is shown at the far right side of the figure; it was generated by (\relation) some illustration software (\activity); which in its turn used data from a composition dataset; the data was generated by composing software; two datasets where used in the process, data set 1 and region list.
The agent executing this process is Derek (\agent), he was associated with the composing and illustration software and is attributed to the creation of the resulting chart.
However, Derek is acting on behalf of the company chart gen (\eg{} he is a contractor or employee).
Both agents in this example have \attributes{} assigned to them to further describe them (\eg{} the full name of the company).



HOW
1 Publications are a common form of representing the provenance of experimental data and results.
1 DOIs are used to cite data used in experiments so that the papers can relate the experimental process and analysis to the actual data used an produced.
1 Several applications of provenance: Data quality (based on the provenance meta data a user can check the quality of data), audit trail (who did what with which data), replication recipes (what was done to get a certain data item), attribution (copyright, ownership, citation, liability), informational (data discovery, context for data).

1 If provenance depends on users manually adding annotations instead of automatically collecting it, the burden on the user may prevent complete provenance from being recorded and available in a machine accessible form that has semantic value.
3 why and where provenance, why is data in the output? where does some data in output come from?
4 In order to minimize its effect on application performance, documentation must be structured so it can be constructed and recorded autonomously by services on a piecemeal basis.
4 three types of assertions: relationship (B was retrieved by applying func1 to A), interaction (received A, sent B), service state (it took 3 seconds to send B after receiving A).