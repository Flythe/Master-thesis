\section{Technical \& Procedural Cornerstones of Security}
\label{security-summarisation}

The outcomes of the interviews were combined with the literature study to compile a practical list with security issues and solutions.
This list is described in table \ref{tab:security-list}, ordered by type and paper.

% Security table with a list of all found problems and solutions in literature and interviews
\begin{center}
	\renewcommand{\arraystretch}{1.3}
	\begin{longtabu}{c X}
		\caption{List of identified risks and solutions, sorted according to type} \label{tab:security-list} \\
		\hline
			\multicolumn{1}{c}{\textbf{Ref.*}}	&	\textbf{Description} \\
		\hline
			\multicolumn{2}{c}{Procedural problems \& risks} \\
		\hline
			\cite{s4Layman2008}, \ref{security-first-interview}	&	Data aggregation and cross referencing makes the identification of individuals possible. \\
			\cite{s18Kum2014}	&	Applying secondary data analysis makes it difficult to cover purpose in the consent process. \\
			\cite{s18Kum2014}	&	Data linkage without identifying data is impossible, data linkage with identifying data is unsafe. \\
			\cite{s18Kum2014}	&	In data linkage privacy has to be temporarily breached in order to identify matching entities in two datasets. \\
			\cite{s18Kum2014}	&	Finding the right amount of identifying data for linkage is a difficult task. \\
			\ref{security-first-interview}	&	Determining identifying data is open to interpretation. \\
		\\ %whitespace
			\multicolumn{2}{c}{Procedural solutions} \\
		\hline
			\cite{s3Herveg2014}	&	Personal data can only be used for the purposes described when the consent was given by the subject. \\
			\cite{s3Herveg2014, s6West2009, s18Kum2014}, \ref{security-first-interview}	&	Data being used should be in a minimum dataset, no superfluous data should be present. \\
			\cite{s3Herveg2014, s15Fenz2014}	&	After the purpose described in the consent has been reached identifying data should be removed from the dataset. \\
			\cite{s3Herveg2014}	&	The purpose of data usage should be: ``specified, explicit, and legitimate''. \\
			\cite{s8FernandezAleman2013}	&	At any point in time and audit of data should be kept, \ie{} provenance of data. \\
			\cite{s8FernandezAleman2013}	&	Accountability is a central part of security. \\
			\cite{s15Fenz2014, s13Patil2014}	&	Apply anonymisation and pseudonymisation to protect identifiable data or to make it impossible to use this data to identify individuals while still being able to use this data for analytical purposes. \\
			\ref{security-first-interview}	&	For each debatable data item describe the purpose it fulfils. \\
		\\ %whitespace
			\multicolumn{2}{c}{Technical solutions} \\
		\hline
			\cite{s6West2009}	&	Use hashes of identifying data to refer to individuals, which is usable for analytical purposes but not for identifying individuals in the real world. \\
			\cite{s6West2009}	&	Store a specified data structure on a secure server. \\
			\cite{s11Rauscher2014}	&	Use ``portholes'' to view data, \ie{} aggregate data for the user to view but do no disclose the dataset. \\
			\cite{s11Rauscher2014}	&	Declassify data when output is requested by the user, hereby removing identifiable data. \\
			\cite{s16Ma2013}	&	Separate identifying data from dataset, in this paper applied to optimise search. Non-identifying data is available for fast search, after making a selection identifying data is appended before outputting. \\
			\cite{s18Kum2014}	&	Using hashes together with (for example) Bloom filters provide a solution to using identifiable data in data linkage. \\
			\ref{security-first-interview}	&	Take note of standard security measures of the present and implement those. \\
	\end{longtabu}
	\par \bigskip
	*: Reference, either refers to a citation (with brackets []) or an interview (paragraph numbering \eg{} 1.2.1)
\end{center}
%PROCEDURAL
%solution
%TODO: 6 Limited dataset under 45 CFR §164.514(e): 'Under certain circumstances, a covered entity may use and disclose protected health information (PHI) in a limited dataset for research, public health, and health care operations purposes. The privacy regulation identifies a list of identifiers that must be removed from data in order for it to be considered a “limited dataset”. Once removed, the information is not deidentified – it is still PHI governed by the privacy regulation. A data use agreement must be signed by those wishing to use limited datasets.'

\section{Analysis}
\label{security-summarisation-subsub}

Some overall points taken from this security analysis will be described and reviewed in the context of the \ivfsystem{}.

Consent is a difficult problem to tackle in research.
When it is required patients need to know what they are signing for and handling data outside of the procedure described is forbidden.
However consent can be avoided by using historical datasets for which it is unreasonable to acquire consent from each patient in them.
This last regulation is what the \ivfsystem{} leans on, which uses historical data.

In order to fulfil regulations and ethical needs a dataset should be minimised so that no superfluous items are left in the dataset.
For each of the data items in the dataset a purpose should be described. 
A proper purpose is leading in ethical discussions about whether to accept a data item in the dataset or not.
Having a well-defined protocol with the \ivfsystem{} can provide more confidence in the system by users.

For data linkage some identifying (\ie{} private) data items are needed.
This can be described in the purpose of the data item, but there are also methods for avoiding these data items.
Hashing of data with the application of Bloom filters make it possible to link two datasets without revealing the identifying data.
Data linkage is only mentioned as a future reference for the \ivfsystem{}, in the first implementation linkage is provided by external sources.

Anonymisation and pseudonymisation should be used to de-identify individuals.
While identification through data aggregation and cross-referencing is still possible to happen these steps should make it more difficult.
The \ivfsystem{} will use both techniques to provide privacy, datasets are mostly kept clean by removing all identifying data.
Whatever identifying data is left (through linkage) will be pseudonymised.

In order to decrease the chances of cross-referencing and data breaches in general provenance of data should be kept.
This means keeping logs on who uses what data at what point in time and what that data looked like at that time.
Apart from privacy this also makes it possible to keep people accountable and to provide data management functionality.
As the \ivfsystem{} is mainly a system to manage data this opens up many implementation possibilities.

Lastly, exploring and using present day standard security measures are a must-have for a good system.
During the software engineering cycle of the \ivfsystem{} searches will be done for the appropriate security measures for each part of the system.
Also the expertise of developers, engineers, and system administrators with multiple years of experience each will be used.