In this chapter the requirement discovery for the \ivfsystem{} will be described.
At the start of the project the assumption was that the system would encompass data management (\eg{} search, select, download) and data analysis (\eg{} support of SPSS or R).
This was however, without any knowledge about the \projectdata{} as it was not available yet.
The dataset is not the scope of this study but a short description is necessary to understand the development process decisions.

\paragraph{\projectdata{} availability}
Data should have been available at the start of the study but it proved to be much more difficult to gather data from the different clinics.
The major part of the problem is the necessity of bureaucracy, as medical data security lies mostly in consent procedures.

In order to evaluate research protocols and data contracts ethical committees are used.
The \project{} involved multiple sites and each of these would only allow data to be released after their own committee gave permission to do so.
Furthermore, the evaluation processes can take quite a long time (up to one year) as some committees only meet a couple of times per year.

On the other side, there are also technical issues.
Early on in the study most of the thirteen clinics had a vendor specific electronic health record (EHR).
Luckily during the study the adoption of a single EHR started to increase, resulting in a mostly standardised data query for a great portion of the clinics.
One other major drawback is the fact that internet is deemed unsafe for data transfers, requiring data gatherers to travel to each of the clinics to physically pick up the data.

Data is not the scope of this study, however the problems encountered with data gathering are.
Chapter \ref{position} goes into a discussion of these problems and will propose solutions to them.

\paragraph{Proceeding development}
The named data issues counted up to a delayed delivery of the data, but also to a delay of the development discussed in this paper.
The data gatherer of the \project{} had to be supported in technical issues as they were not equipped with the required expertise.
Providing this support took time, but the lack of data also meant that developing the system ran into some problems.

Bringing this system's concept into a brainstorm session proved to be quite difficult, without experience with the data it was too abstract for the users to form useful ideas.
Therefore, a study was performed to find potential requirements and further define the system's description (\ie{} make it less abstract).
This study consisted of literature studies, an interview, and observations.

The literature study, resulted in descriptions of security issues and solutions.
These are the underlying requirements of the \ivfsystem{} and have to be implemented as a result of the sensitive data repository.
The interview tied abstract security concepts together with real-life situations.
And lastly observations led to the concept of the work process that had to be supported. 

With this input an initial requirement analysis was created which in its place was used as input for a brainstorm session.
The results of this session were then used to update the requirements to form the final concept.