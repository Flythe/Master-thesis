In this chapter the requirement discovery for the \ivfsystem{} will be described.
At the start of the project the assumption was that the system would encompass data management (\eg{} search, select, download) and data analysis (\eg{} support of SPSS, SAS, or R).
This was, however, defined without any knowledge about the \projectdata{}, as it was not available yet.
The dataset is not the scope of this study, but a short description is necessary to understand the development process decisions.

\paragraph{\projectdata{} availability}
Data should have been available at the start of the study, but it proved to be much more difficult to gather data from the different fertility clinics than anticipated.
The major problem is the necessity of a strict data delivery protocol, as medical data security lies mostly in consent procedures.

Ethical approval was the first barrier.
Ethical committees have the task to evaluate research protocols and data exchange contracts.
\project{} involved multiple sites and each of these would only allow data to be released after their own committee gave permission to do so.
Furthermore, the evaluation processes can take quite a long time (up to one year) as some committees only meet a couple of times per year.

Later on there were also technical aspects to solve.
Early on in the study most of the thirteen clinics had a vendor-specific electronic health record (EHR).
Luckily during the study the adoption of a single EHR started to increase, resulting in a mostly standardised data query for a great portion of the clinics.
One other major drawback is the fact that internet is deemed unsafe for data transfers, requiring data gatherers to travel to each of the clinics to physically pick up the data.

\paragraph{Approach adopted in this study}
The difficulties above caused a delayed delivery of the data.
Moreover, the data gatherer of \project{} had to be supported in technical issues, as she was not equipped with the required (technical) expertise. 
Providing this support took time, but the lack of data also caused a delay in the \ivfsystem{} development.

Bringing the \ivfsystem{} concept into a brainstorm session proved to be quite difficult. 
Without experience with the data, it was too abstract for the users to form useful ideas.
Therefore, a study was performed to find potential requirements and further define the system's description (\ie{} make it less abstract).
This study consisted of literature studies and an interview (section \ref{security}), and observations (section \ref{process-analysis}).

The literature study resulted in descriptions of security issues and solutions.
These are the underlying requirements of the \ivfsystem{} and have to be implemented as a result of the sensitive data repository.
The interview with an expert tied abstract security concepts together with real-life situations.
And lastly observations led to the concept of the work process that had to be supported. 

With this input an initial requirement analysis was created, which was used as input for a brainstorm session (section \ref{brainstorm}).
The results of this session were then used to update the requirements to form the final concept.