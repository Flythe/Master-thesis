\section{Security}
\label{security}

Earlier it was noted that security is an integral part of the \ivfsystem{}, this section contains the justification for this statement.
The full security review can be found in appendix \ref{security-review-appendix}, the following section contains the analysis when considering the found security aspects within the context of the \ivfsystem{}.

\subsection{Security Analysis: the \ivfsystem{} case}
\label{security-summarisation-analysis}

Points taken from this security study will be described and reviewed here in the context of the \ivfsystem{}.

Starting with consent, for the \project{} it can be viewed from multiple perspectives: patient, clinic, and registry.
When a researcher wants to use the dataset available in the \ivfsystem{}, they will use data coming from the clinics, which in turn gather data from patients.
This patient data is then linked to the PRN registry data.
Each of the parties involved should to some extent be able to determine if they allow their data to be used.

Patient consent is a difficult problem to tackle in research in general.
When giving consent, patients need to know what they are signing for and handling data outside of the goal which was described is forbidden.
However, when using datasets for which it is unreachable and unreasonable to acquire consent from each patient in them, there are exceptions in the Dutch consent regulations.
This exception is what the \ivfsystem{} currently leans on. 
It uses historical data for the years 2000 to 2010 and according to the nationwide IVF report \cite{ivfReportNVOG} there are approximately 4000 pregnancies per year.
Which means that there are about 40.000 patients in the dataset in total.
Given the size and age of the dataset it was deemed unreasonable to require consent.
To determine if consent is not a requirement, advice from external parties should be acquired.
In this case these were: the \AMC{} chief privacy officer, medical ethical commissions of data suppliers, and the \PRN{} privacy commission.

Consent from clinics and registries can be compared to patient consent.
They all give permission to use \emph{their} data for a specific cause as described in the consent.
The main difference between these data providers in giving consent is that their considerations are based on different interests.
For example, a patient might be concerned about his/her privacy.
Of course a clinic will also take this into account when a dataset is requested but they also have interests like: what research will be performed with the data.
If this clashes with a research of their own it is less likely the clinic will give consent.
In the \ivfsystem{} these different levels of consent must be taken into account  to be able to perform the function of providing research data.

In order to fulfil regulations and ethical needs a dataset should be minimised, so that no superfluous items are left in the dataset.
For each of the data items in the dataset a purpose should be described. 
A proper purpose is leading in ethical discussions about whether to accept a data item in the dataset or not.
Having a well-defined protocol with the \ivfsystem{} can provide more confidence in the system by users, leads to better understanding of the system, and provides evidence that choices about data items were made with certain considerations.

For data linkage some identifying (\ie{} private) data items are needed.
This can be described in the purpose of the data item, but there are also methods for avoiding these data items.
Hashing of data with the application of Bloom filters \cite{something} makes it possible to link two datasets without revealing the identifying data.
On-line data linkage is only mentioned as a future work for the \ivfsystem{}.
In the first implementation, linkage is provided by a third-party and the linked data itself is seen as an external service in figure xx.

Anonymisation and pseudonymisation should be used to de-identify individuals.
While identification through data aggregation and cross-referencing is still possible to happen, these steps should make it more difficult.
The \ivfsystem{} will use both techniques to provide privacy, datasets are mostly kept clean by removing all identifying data at the data gathering step.
Whatever identifying data is left (through linkage) will be pseudonymised before it is accepted into the system.

In order to decrease the chances of cross-referencing and data breaches in general, auditing should be applied.
This means keeping logs on who uses what data at what point in time and \silvia{?? what that data looked like at that time}.
Apart from privacy this also makes it possible to keep people accountable and to provide research data management functionalities such as archival and provenance.

Lastly, exploring and using present day standard security measures are a must-have for a good system.
During the software engineering cycle of the \ivfsystem{} searches will be done for the appropriate security measures for each part of the system.
Also the expertise of developers, engineers, and system administrators with multiple years of experience each will be used.