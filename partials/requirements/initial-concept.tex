\paragraph{Initial concept}
Before developing software the process has to be defined in terms of functions; this is the system's concept.
This section describes how the results of the process analysis were transformed to an initial (\ie{} pre-brainstorm) concept.

The initial concept defined a system for the \projectdata{} with capability for data request, management, analysis, and security.
Educated guesses were made to find all the parts needed to support these requirements.
Figure \ref{fig:brainstorm-before} describes the full view of the initial concept. 
The function groups presented in figure \ref{fig:research-workflow} are expanded into: users, external components, data, and functions.

Two direct users and several external users are planned, each with their specific set of functions with the data they use and produce.
Additionally, external components such as linked and unlinked data, committee protocols, and data administration personnel had to be described.
These components are essential parts that are pre-configured into the system, but considered outside of the scope of this study.

Data organisation was planned as follows: the \projectdata{} contains linked fertility clinic and \PRN{} data, but also unlinked data where no match could be found.
This data is the `raw' data of the system.
Raw data may be grouped into subsets that can be analysed, resulting in analysis outcomes.
Metadata is used to describe or annotate raw data, which can add meaning or extra information (\eg{} date, file format, etc.).
Provenance and audit data are a result of security measures; this data is generated by the system and can be used by the data manager to perform security tasks.

Lastly, note that a data request is formulated on the system, but the actual approval happens outside of the system.

\begin{figure}[h]
	\centering
	\includegraphics[width=1.0\linewidth]{images/brainstorm-before}
	\caption{
		Initial concept for the \ivfsystem{}, encompassing data and user management. 
		The system offers different sets of functions for three user roles (researcher, data manager, and interested third parties) indicated by colours. 
		External components are (offline) essential parts for system (\eg{} data, regulations) but are outside the scope of development.
		Data listed is either available at initialisation of the system or is generated during execution.
	}
	\label{fig:brainstorm-before}
\end{figure}