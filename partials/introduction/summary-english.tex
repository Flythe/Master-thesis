\clearpage
\phantomsection
\addcontentsline{toc}{section}{Abstract}

\section*{Abstract}

In the Netherlands about 5\% to 8\% of all couples remain childless due to infertility or subfertility. 
Several treatments may be used to assist in reproduction. 
However, outcome indicators related to the (born) child are relatively unknown. 
In order to find out a study (\project{}) was started to gather and link data from both the fertility clinics and the national birth registry. 
This data is captured in the \projectdata{}.

This study describes the development of the \ivfsystem{} and investigates attitudes towards medical (big) data usage. 
The \ivfsystem{} was initially meant to support researchers with data management of the \projectdata{}, \eg{} querying or analysis. 
From a requirement study an initial concept was described, which was used as input for a brainstorm session with stakeholders.
The session showed that apart from data management the \ivfsystem{} should encompass many more aspects, mostly concerning data reuse and aimed at data owners. 

An in-house project was chosen as a development starting point for the \ivfsystem{}. 
With minimal changes to the data model it was possible to integrate the \projectdata{}.

During evaluation of the prototype it became clear that more polishing of the system's workflow (i.e., the manner in which a user moves through the system) and several iterations of user-centred design are needed. 
Furthermore, more functionality and security should be added to the system.

In this article three main concepts are discussed: experiences and impressions during the data gathering process, the focus switch of the \ivfsystem{} from a data management to a data reuse supporting system, and possible extrapolation to other data domains. 

\paragraph{Keywords}
Software engineering, data reuse, data gathering, request management, medical domain
