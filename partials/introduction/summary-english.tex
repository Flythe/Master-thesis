\clearpage
\phantomsection
\addcontentsline{toc}{section}{Abstract}

\section*{Abstract}

In the Netherlands about 5\% to 8\% of all couples remain childless due to infertility or subfertility.
Several treatments are used to assist in reproduction.
However, it is relatively unknown what the outcomes are for the born children.
The \project{} was started to gather and analyse data from both the fertility clinics as the national birth registry (\ie{} \PRN{}).

The idea for the \ivfsystem{} was to support data management, \eg{} querying or analysis.
First a requirement study was performed. 
An initial concept was described and used as input for a brainstorm session with stakeholders.
The session showed that there are many more aspects for the system, mostly concerning data reuse.

An in-house project was chosen as a development starting point for the \ivfsystem{}.
With minimal changes to the data model, it was possible to integrate the \projectdata{}.

During the evaluation of the prototype with users it became clear that more polishing of the system's workflow (\ie{} the manner in which a user moves through the system) and one or two iterations of user-centred design are needed.
Furthermore, missing functionality and security should be added to the system.

Discussed are the experiences and impressions during the gathering data process, the focus switch of the system from a data management to a data reuse supporting system, and possible extrapolation to other data domains.

\paragraph{Keywords}
Software engineering, data reuse, data gathering, request management, medical domain