\paragraph{The domain and background}
Reproduction is a fundamental building block of life.
For the human species this means that two individuals, with a different sex each, produce offspring.
The offspring contains the genetic material of both the parents.
However, there are many conditions and diseases which can lead to infertility or subfertility.
In the Netherlands these terms are defined in a national guideline by the Dutch association of obstetrics and gynaecology (NVOG\footnote{Dutch: Nederlandse Vereniging voor Obsetrie en Gynaecologie}) \cite{subfertilityGuideline}.
Infertility is defined as a rare condition where ``no chance of reproduction exists'', 
and subfertility as ``failure to become pregnant after twelve months of unprotected coitus aimed at conception''.
Approximately 5\% to 8\% of all couples in the Netherlands remain without children unwillingly \cite{cbsStatistics, nhgStatistics}.

Luckily there are several fertility treatments.
Some of these lead to both the parents becoming biological parents. 
Others make use of donor material or surrogates, thus the child does not contain the genetic material of one of the `parents'.
Commonly used treatments include intrauterine insemination (IUI) and in vitro fertilisation (IVF) \cite{treatmentExplanation}.
%Of these IUI is closer to regular impregnation, as sperm cells are collected and injected deep in the uterus, while IVF happens `in vitro', \ie{} in glass, outside of the body.
%IVF can be further divided into more specific treatment types (\eg{} intracytoplasmic sperm injection, ICSI), 
%the difference between these being the used technique or the type of parental materials used.
%For example, there are fresh and frozen treatments; in a frozen treatment, material of the male or female is kept in (frozen) storage before being used.
%Nevertheless, each treatment follows about the same steps: egg maturation stimulation, egg retrieval, fertilisation, and embryo transfer \cite{treatmentExplanation}.
Each treatment follows about the same steps: egg maturation stimulation, egg retrieval, fertilisation, and embryo transfer \cite{treatmentExplanation}.
The stimulation phase can also be called the start of a new \emph{cycle}. In The Netherlands (according to the NVOG) 14,562 of these cycli were started in 2013 \cite{ivfReportNVOG2013}, approximately 30\% of which resulted in a ongoing pregnancy.
The success rate for a given clinic or treatment is fairly well known.
However, outcome quality indicators related to the (born) child are either sparse or unknown.

All perinatal data in the Netherlands have to be entered into the perinatal registry (perinatale registratie Nederland, \PRN{}\footnote{http://www.perinatreg.nl}).
The registry exists of population based data on pregnancies, provided care, deliveries and (re)admissions of newborns.
For research purposes, however, this data is completely separated from the clinic's patient data.
The Dutch healthcare system is quite exceptional as fertility clinics are in the public domain, thus there is pressure for disclosing  data for research and governance reasons.

With minimal identifying data from both the fertility clinics and the \PRN{}, treatment input and outcome can be linked together.
To execute this linkage the Dutch Assisted Reproductive Technology Study (\project{}) was established.
During the project, data between 1999 and 2010 is gathered from each of the thirteen Dutch fertility clinics and linked to the \PRN{}.
This fertility data covers only ongoing pregnancies, as a child has to be born in order to link to the \PRN{}.
In the given period of time, about 44,164 ongoing pregnancies were registered in the clinic datasets \cite{ivfReportNVOG}.
Linkage will inevitably result in a loss of a few percent where no appropriate match can be found, 
but a considerate amount of pregnancies is available for research.
The linkage part of the research is outside of the scope of this paper and will be provided by another project \cite{}.
It can be argued that the available data can be seen as big data, this will be described in the following section.