\clearpage
\phantomsection
\addcontentsline{toc}{section}{Samenvatting (Dutch summary)}

\section*{Samenvatting}

In Nederland blijft tussen de 5\% en 8\% van alle stellen ongewenst kinderloos als gevolg van infertiliteit of subfertiliteit.
Een aantal vruchtbaarheidsbehandelingen kan worden toegepast, maar het is nog relatief onbekend wat de uitkomsten hiervan zijn voor de geboren kinderen.
Om dit te bestuderen is de studie DARTS! gestart, met als doel om data van fertiliteitsklinieken te verzamelen en te koppelen aan het nationale geboorteregister.
Deze data is vastgelegd in de \projectdata{}.

Deze studie beschrijft de ontwikkeling van het \ivfsystem{} en onderzoekt daarnaast de houding ten opzichte van het gebruik van medische (big) data in het algemeen.
In eerste instatie was het idee voor de \ivfsystem{} om onderzoekers te ondersteunen met datamanagement van de \projectdata{}, bijvoorbeeld querying en analyse.
Met behulp van een requirement studie werd een initieel concept beschreven, welke werd gebruikt als input voor een brainstormsessie met belanghebbenden.
De sessie maakte duidelijk dat, naast datamanagement, het \ivfsystem{} meerdere aspecten moest bevatten.
Waarbij de belangrijkste aspecten zijn: het hergebruik van data en de eigenaren van deze data.

Een intern project werd gekozen als startpunt voor de ontwikkling van het \ivfsystem{}.
Met minimale aanpassingen aan het datamodel was het mogelijk om de \projectdata{} te integreren.

Tijdens de evaluatie van het prototype werd duidelijk dat de workflow van het systeem verbeterd moet worden en dat verbetering van de gebruikersomgeving nodig is.
Verder moet er meer functionaliteit en beveiliging toegevoegd worden aan het systeem.

In dit artikel worden drie hoofdconcepten bediscussieerd: ervaringen en impressies gedurende het data verzameling proces, de focusverandering van het \ivfsystem{} van datamanagement naar data-hergebruik en mogelijke extrapolatie naar andere datadomeinen.