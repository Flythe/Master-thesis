\phantomsection
\addcontentsline{toc}{section}{Samenvatting (Dutch summary)}

\section*{Samenvatting}

In Nederland blijven tussen de 5\% en 8\% van alle stellen ongewenst kinderloos door infertiliteit of subfertiliteit.
Een aantal vruchtbaarheidsbehandelingen worden toegepast.
Maar het is relatief onbekend wat de uitkomsten zijn voor geboren kinderen.
Het \project{} is gestart om data te verzamelen en analyseren van fertiliteitsklinieken en het nationale geboorte register.

Het idee voor de \ivfsystem{} was om data management te ondersteunen, bijvoorbeeld querying en analyse.
Een requirement studie werd gedaan en een initieel concept werd beschreven.
Dit concept werd gebruikt als input voor een brainstormsessie met belanghebbende.
De sessie maakte duidelijk dat er meerdere aspecten zijn aan het systeem, vooral data hergebruik.

Een intern project werd gekozen als ontwikkel startpunt voor het \ivfsystem{}.
Met minimale aanpassingen aan het data model was het mogelijk om de \projectdata{} te integreren.

Tijdens de evaluatie van het prototype met gebruikers werd het duidelijk dat: de workflow van het systeem (de manier waarop een gebruiker zich door het systeem heen beweegt) verbeterd moet worden en dat een of twee iteraties van user-centred design nodig zijn.
Verder moeten missende functionaliteit en beveiliging toegevoegd worden aan het systeem.

Bediscussieerd worden: ervaringen en impressies gedurende het data verzameling proces, de focus verandering van het systeem van data management naar data hergebruik en mogelijke extrapolatie naar andere data domeinen.