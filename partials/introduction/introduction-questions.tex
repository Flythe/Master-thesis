\paragraph{Using IT as leverage}
Summarising, the challenges come down to a change of attitude.
Even though literature describes big data as a benefit for the users, medical researchers are shying away from using it.
How can they be convinced that following certain big data guidelines can evolve performing research itself?

This work shows a proposal for a supportive system which can manage data produced by the \project{}; its working name is: \project{} Research Gateway (\ivfsystem{}).
It is meant to show what value can be delivered if some human performed functions are left for a computerised system in the management of such a valuable and sensitive dataset.
In order to give direction to the development, the following main aspects have been investigated: security, data access, data browsing, and data querying.
This resulted in the following research questions:

\begin{enumerate}
	\item How do we implement a user-friendly system in a \IVF{}--\PRN{} medical domain which covers problems concerning: data security, data access, data browsing, and data querying?
	\item What needs to be changed in the current attitude towards data usage to promote big data in a \IVF{}--\PRN{} medical domain?
\end{enumerate}

These two questions were broken down into sub questions. 
For question 1:
\begin{itemize}
	\item What are the functions of this system and which parts of the research process should this system support?
	\item Who are the users and what are the use cases for these users?
	\item What are the legal and security aspects of this system?
	\item What is the data model for this system?
	\item What is the minimum prototype demonstrating that the system's goals are reachable?
	\item To what extent does this system meet the expectations of users?
\end{itemize}

For question 2:
\begin{itemize}
	\item What are the promoting aspects of data usage?
	\item What are the blocking aspects of data usage?
	\item What alignment needs to take place to promote data usage?
	\item How can IT be leveraged to achieve this goal?
\end{itemize}