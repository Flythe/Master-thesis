\silvia{how will you give shape to this paper? will it be based on your own observations (during this study) or citing a lot of literature, or both?}

\section{Introduction}
\paragraph{Problem}
Describe data `usage' phases and what the problems are now:
\begin{itemize}
	\item Study preparation, data usage approval
	\item Data retrieval
	\item Data preparation
	\item Data analysis
	\item Data reuse
\end{itemize}

\silvia{where is data collection and credit for data reuse, provenance? }

\paragraph{Position}
There should be more data re-use and cooperation between data producers/consumers.
IT can help with the alignment of these two factions.


%the lack of time to implement the prototype is not well explained. 
%i think somehwere in your thesis you need to explain the data collection to build the repository. 
%the different formats, difficulty to access, etc. 
%this was the reason why you did not have timein the end to do what we planned - because the data was not there, which would make the whole requirement analysis exercise too abstract and futile.
%no data
%difficult to gather, different formats, difficult to access
%social problems, regulation problems, technical problems (in that order)
%first the heads of clinics need to be in line, then the clinicians
%then everything needs to be aligned with Dutch regulations (it is actually intertwined with the previous one as good protocols will lead to better understanding and more likelihood that they will agree).
%technical problems, once everyone agrees we should do this and it is allowed we run into the next problem, it is hard to gather the data