\section{Checklist A}
\label{security-checklists-a}

This section gives two checklists found in the literature.
These lists give a number of points which a system should comply to in order to be secure in the sense of privacy.
The first list describes items used to test safety of the implementation of a patient-centred eHealth solution in Dehling \cite{s17Dehling2014}:
\begin{itemize}
	\item No unauthorized person must be able to access patients' information;
	\item The real identity of patients must not be revealed;
	\item Access must be limited to necessary information and data segregation must be ensured;
	\item Unnecessary access rights must be revoked;
	\item It cannot be possible to force patients to reveal information they do not want to reveal;
	\item Eavesdropping has to be prevented during transmission and storage;
	\item It must not be possible to reveal relationships between items through observation;
	\item It must be ensured that information content is as intended and not unintentionally changed;
	\item Up-to-date information must be available whenever needed;
	\item Redundancy must be employed to ensure that data can be restored;
	\item It must be possible to store information as long as it is required (even a lifetime or longer);
	\item It must be possible to restore lost information to a specific point in time;
	\item Failure of single nodes must not impede the performance of the whole service;
	\item Systems have to be adaptable to changing performance needs;
	\item There cannot be a significant delay between data entry and dissemination to patients;
	\item Accesses to and uses of information must be attributed to the respective party and it must not be possible to deny such actions afterwards;
	\item Relevant activity (e.g. document accesses) must be logged;
	\item It must be determined who is using the software and verified that they are who they claim to be;
	\item The boundaries of trusted access to the information system must be known and controlled;
	\item Unintended actions and/or activity must be detected;
	\item Unauthorized access must be avoided and access rights must be managed;
	\item Impairment of hardware (theft, natural disasters, ...) has to be prevented;
	\item System vulnerabilities must be detected;
	\item Important information has to be easily accessible;
	\item Patients have to be able to control who can access what information;
	\item Authorization details must be substitutable (loss, technological obsolescence);
	\item User ethics, obligations, and proficiency must be reinforced;
	\item In case of emergency, medical professionals must be able to access required information;
	\item Patients have to agree to uses of their information and patient consent must be managed;
	\item Patients have to be able to retrieve information stored on them.
\end{itemize}

\section{Checklist B}
\label{security-checklists-b}

The second list describes questions used to test EHR systems on security and privacy in Fernández-Alemán \cite{s8FernandezAleman2013}:
\begin{itemize}
	\item What standards and regulations does the system satisfy?
	\item Does the system use pseudo anonymity techniques?
	\item Is the user data encrypted? 
	\item What authentication systems are used? 
	\item Can access policies be overridden in the case of an emergency? 
	\item If the system needs user roles, who defines them? 
	\item Who grants the access to the data? 
	\item What kind of information is exchanged? 
	\item Are there audit logs?
	\item Are the systems' users trained in security and privacy issues? 
\end{itemize}