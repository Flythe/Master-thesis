\section{Implementation}

% Functions as implemented
% Architecture as implemented
% Datamodel as implemented
% Design as implemented

\silvia{instead of presenting what has been done, this section is about what has not been done. it does not seem useful to spend so much space on describing things that were not done.
please make sure to describe the complete story (all requirements and all functions) in one place, and mark those that were not included in the prototype. the discussion about what was included and what was not can be part of discussion and conclusions - and not mixed in the implementation description. here you should clearly present what have been done, with technical details but with scientific language (methods, results, evaluation). }
%\paragraph{Division of management tasks}
\silvia{too long - just explain the prototype implementation here. it is not necesary to explain the dev along time, but just how the prototype became in the end}
Due to time restrictions in this study not all functions that were discovered in chapter 2 can be implemented.
A selection is made based on the programmers opinion what would be most profitable for a prototype system.
The selection that is made can be seen in figure \ref{fig:functions-implemented}.
Decisions are based on the fact that the demo has to appeal to a wide variety of users, most importantly research and clinic management.

The most basic system data management functions are implemented, namely 
search and filtering for the researcher and data manager roles, and the download function.
Data audit is implemented through placeholders to show the value of such an function without the actual data processing back-end. \silvia{what do you mean by placeholders (my spelling says this word does not exist)? is this fake functions with pre-defined responses? or a no-op?}
\silvia{??Metadata handling is not implemented, as data is considered \emph{fixed} after system initialisation.}

After this, request management is implemented.
Removing direct access to data and adding more control for committee users (coming directly from each clinic) is deemed to increase system trust by clinic management.
Request creation, editing, and submission by researcher users is supported.
Approval by committee users and automated subset creation by the system is also implemented.
Not implemented are: annotation and feedback on requests, stale request detection, and keeping provenance.
All of these functions are mostly supporting users in performing tasks but are not critical to the execution of the system.

User management is implemented with the addition of roles to user accounts and a dashboard for the data manager user.
In this dashboard the manager can find all the users of the system and assign the appropriate roles to each of them.
Lastly, publication management as closing stone of the research cycle is left out.
At this moment the additional time that has to be spend including these functions does not outweigh the added value for prospective demo users.

\paragraph{Security}
Even though security considerations are a big part of the requirement analysis (see section \ref{security}), it does not show itself that clearly in the system implementation.
Most of the security measures were taken during the data gathering steps.
Because the decision was made to have a fixed dataset for the system a lot of the discussed security measures do not need to apply anymore as described in section \ref{security-summarisation-analysis}.

Provenance, as part of security, was not implemented either.
It would be very useful to show the strength of the \ivfsystem{}, data protection through provenance can be made highly visible and easy to understand for humans.
However, the goals to quickly support the provenance functions and develop an interface on top of the data could not be reached within the time frame.

\silvia{move this to discussion}
Lastly, in order to obtain a working system a few more things are needed.
More functionality needs to be implemented, most notably better user interfaces for data filtering and selection (see section \ref{evaluation}).
Furthermore, more design and programming iterations need to be done before the request and publication management have been fully developed.
Then a big security iteration should be done to prepare the system for testing and certification.
Right now security has been patched up in the front-end by hiding information for certain users, this should change to the back-end not \emph{providing} that information to begin with.

