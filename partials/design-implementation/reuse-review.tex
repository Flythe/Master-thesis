\section{Technical Design Considerations}
\label{reuse}

%\subsection{Review}
%\label{reuse-review}

The data management
is the most significant part of the \ivfsystem{} and therefore should be well implemented.
Also implementation had to be done in a short time due to study planning restrictions.

Multiple systems were considered and evaluated for reuse.
Because extensions would be needed, the system preferably had to be open-source.
Three systems were included in an more in-depth evaluation presented below.

\silvia{put this info below in the right places}
two externally developed open-source systems and one in-house project.
The external software was identified through the paper of Leroux \cite{leroux2011}.

\paragraph{External systems.}
The external systems identified were OpenClinica and DADOS Prospective, 
which are both open-source clinical trial management systems.
OpenClinica \cite{} is ... - an online demo is available \cite{}.
DADOS Prospective \cite{} is .....
\silvia{maybe here you could list others that you have looked into, but that were no adequate? only names and references are ok. BTW: references to scientific pubs should be preferred to urls in all cases}

Their focus lies on data entry and retrieval for low-level (researcher) users, and on research overview for high-level (management) users.
Besides data entry, they also offer overviews displaying statistics on participants and the clinics they belong to, how many inclusions were made, follow-up percentages, etc.

Data collection protocols can be defined per study in a very flexible manner.
After the protocol has been defined, it is fixed for all study participants.
This is very useful in longer running (prospective) studies, where collection should be standardised for quality, analysis, and reuse purposes.
In principle, as far as notable for OpenClinica, there are no features directly supporting data sharing or reuse.
\silvia{this cannot be true. there is no concept of external people sharing data, but if people are made members of the study they can get access to the whole dataset. this is a different model than we want for our project, where there is no concept of study to create compartments for data and access control. please reformulate, better explaining the difference between a study-per-study recording system, as opposed to a registry of data that will serve external, ad-hoc data requests? }

\paragraph{In-house project: Rosemary.}
\silvia{i think this can be improved. highlight metadata search functions}
A system named Rosemary was built at the AMC to handle neuroimaging data and metadata, data analysis applications, and their execution on grid infrastructures.
The data are brain images generated by Magnetic Resonance Imaging (MRI) scanners, and the 
metadata refers to the subjects, the imaging session, etc. 
These images are imported into the system from external data servers (XNAT\cite{}), selected by the user, and submitted for processing by analysis applications. Result are also stored in the system and used in further analysis.
Input is restricted to automated functions and there are no manual input interfaces available.
However, once data is in the system it allows for each data item to be used in access restricted subsets.
This system data has interesting characteristics for metadata searching and ... ?

\paragraph{The better pick.}
It quickly becomes apparent that no existing system is able to cover all the needed functionality, even when looking at only data management. \silvia{this sounds a bit out-of-the-blue without a clear definition of the functions you are after, and a consistent check for each one of the functions in the existing systems. if you would make a table it would be very clear to see that the systems do not fulfil, and which functions would be necessary to add in our prototype}
Based on several considerations, the decision was made to use the in-house Rosemary project for further development
Flexibility in the code is important, \ie{} open-source, extensible, easy to develop. \silvia{but all 3 are open source, so no difference here}
Next to this, short lines to expertise are useful for a quick development process, because once a (coding) problem is encountered it can be solved in a matter of hours.
Lastly, during the evaluation process it became clear that the data model of Rosemary could be reused with minimal effort, \silvia{because... whereas the others ...}.

\silvia{this part was in next section, incorporate here somehow}

%\paragraph{Goodness of fit}
\silvia{i have the feeling that this is still part of the previous section, to explain the choice. }
To make clear how well Rosemary can be fitted to the \project{} first the functions of the current implementation will be described.
As said in the previous section the system encompasses domain specific data and processing management. \silvia{you can forget processing management, right? focus focus focus ;-)}
The shear amount of available data and metadata is what makes the data management in this case a challenge.

Data management challenges are tackled by providing extensive search, filter, and selection functionality.
Submission management is handled by automatically bundling submissions into processings which are then fed to one or multiple applications.
Furthermore, notifications and messaging are supported.
Notifications provide users with system status information. 
For example, progress updates during a data import or when receiving a message.
Messaging should enable collaborative work by wrapping research communication inside the same system as where the data resides.
