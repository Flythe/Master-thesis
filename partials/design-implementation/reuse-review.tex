\section{Re-use}
\label{reuse}

\subsection{Review}
\label{reuse-review}

For the reuse portion of this system the main criteria was data management.
This is the most significant part of the \ivfsystem{} and therefore should be well implemented.
Also implementation had to be done in a short time due to study planning restrictions.

Multiple systems were considered and evaluated.
Because extensions were needed the system preferably had to be open-source.
Three systems were included in an more in-depth evaluation, two externally developed open-source systems and one in-house project.
The external software was identified through the paper of Leroux \cite{leroux2011}.

\paragraph{External system evaluations}
The external systems identified were OpenClinica and DADOS Prospective.
Both are open-source and for OpenClinica an online demo is available.
These software packages are called clinical trial management systems, which quickly becomes apparent when using the system.
Focus lies on data entry and retrieval for low-level (researcher) users, and on research overview for high-level (management) users.
Overview meaning displaying statistics on participants and the clinics they belong to, how many inclusions were made, follow-up percentages, etc.

Per study data collection protocols can be defined, when starting a new study the definition functionality is very flexible.
After the protocol has been defined it is fixed for each of the study participants.
This is very useful in longer running (prospective) studies where collection should be standardised for quality, analysis, and reuse purposes.
In principle, as far as notable for OpenClinica, there are no features directly supporting data sharing or reuse.

\paragraph{In-house development}
A system named Rosemary has been developed in-house at the same department this study was performed at.
Handling data, applications, submissions, and input and output of these submissions is what it was build for.
The context domain is neuroscience, the data used comes from MRI machines and are mainly references to images and their metadata.
These are then submitted for processing by applications resulting in outcomes (also stored in the system) used in further analysis.

Considering the data management part, there are data input, filtering, and reuse possibilities.
Input is restricted to automated functions and there are no manual input interfaces available.
However, once data is in the system it allows for each data item to be used in access restricted subsets.

\paragraph{The better pick}
It quickly becomes apparent that no existing system is able to cover all the needed functionality, even when looking at only data management.
The decision was made to use the in-house Rosemary project for further development.
The choice is made based on several considerations.
Flexibility in the code is an important item \ie{} open-source, extensible, easy to develop.
Next to this short lines on expertise are useful for a quick development process, once a (coding) problem is encountered it can be solved in a matter of hours.
Lastly, during the evaluation process it became clear that the data model could be reused with minimal effort.