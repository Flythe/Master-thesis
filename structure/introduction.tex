2 pages

\paragraph{The domain and background}
What is the clinical background of the problem. 
Outcome of fertility treatment is unknown.
Solved by linking PRN to IVF (introduce \project{})?
\paragraph{What is big data?}
\allard{(Where do I include a small piece of base information about big data?)}
Why is big data of importance for this problem?
Introduce that data is being collected and should be shared/used in research.
Move towards the point that there is a problem with data processes.
\paragraph{The data problem}
Talk about the research type in which data is being used (registration), say how other types of research (trials, RCT) might also benefit in the end.
So what is the actual (data) problem we have?
\paragraph{Using IT as leverage}
Propose that we use IT to overcome the problem.
How should the system help with the problem (link back to big data problems)?
\paragraph{Research questions}
\begin{itemize}
	\item How do we implement a user-friendly system in a IVF medical domain which covers problems concerning: data security, data access, data browsing, and data querying?
	\begin{itemize}
		\item What are the legal and security aspects of this system?
		\item What is the data model for this system?
		\item What are the functions of this system and which parts of the research process should this system support?
		\item Who are the users and what are the use-cases for these users?
		\item What functions were actually implemented in the prototype?
		\item To what extent does this system meet the expectations of users?
	\end{itemize}
	\item What needs to be changed in the current attitude towards data usage to promote big data in a IVF medical domain?
	\begin{itemize}
		\item What are the blocking aspects of data usage?
		\item What are the promoting aspects of data usage?
		\item What alignment needs to take place to promote data usage?
		\item How can IT be leveraged to achieve this goal?
	\end{itemize}
\end{itemize}