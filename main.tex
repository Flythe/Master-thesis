\documentclass[a4paper]{report}

% Packages
\usepackage{enumitem}
\usepackage{graphicx}
\usepackage{array}
\usepackage{float}
\usepackage{url}
\usepackage[maxbibnames=5,style=numeric,backend=bibtex,sorting=nty,firstinits=true]{biblatex} \bibliography{ref}
\usepackage[noabbrev,capitalise]{cleveref} 
\usepackage{tabularx}
\usepackage{tabu}
\usepackage{longtable}
\usepackage[british]{babel}
\usepackage[font=footnotesize]{caption}
\usepackage{csquotes}
\usepackage{color}
\usepackage{rotating}

% Settings
% Margins in lists
\setlist[itemize]{noitemsep}
\setlist[enumerate]{noitemsep}
% Margins in tables
\renewcommand{\arraystretch}{1.3}

% Text replacement commands
\newcommand{\ie}{i.e.,}
\newcommand{\eg}{e.g.,}
\newcommand{\IVF}{IVF}
\newcommand{\PRN}{PRN}
\newcommand{\AMC}{AMC}
\newcommand{\project}{DARTS!}
\newcommand{\projectdata}{D-dataset}
\newcommand{\ivfsystem}{D-gateway}
\newcommand{\escience}{e-Science}

% Commenting commands
\newcommand{\silvia}[1]{\textcolor{red}{\textbf{*Silvia: }\textit{#1}}}
\newcommand{\allard}[1]{\textcolor{magenta}{\textbf{*Allard: }\textit{#1}}}
\newcommand{\note}[1]{\textcolor{cyan}{\textit{#1}}}

%\title{Developing a Data Management System for Obstetric Research Into Long Term Effect of In Vitro Fertilisation Techniques}
\title{A Data Management System for Obstetric Research}

\author{
	Allard J. van Altena\\
	University of Amsterdam\\
	Email: a.j.vanaltena@amc.uva.nl
}

\begin{document}

%	\maketitle
	
	\tableofcontents
	
	\chapter{Introduction}
	\label{introduction}
	
	\paragraph{The domain and background}
Reproduction is a fundamental building block of life.
For the human species this means that two individuals, with a different sex each, produce offspring.
The offspring contains the genetic material of both the parents.
However, there are many conditions and diseases which can lead to infertility or subfertility.
In the Netherlands these terms are defined in a national guideline by the Dutch association of obstetrics and gynaecology (NVOG)\footnote{Dutch: Nederlandse vereniging voor obsetrie en gynaecologie}\cite{subfertilityGuideline}.
Infertility being defined as a rare condition where ``no chance of reproduction exists''.
And subfertility as ``failure to become pregnant after twelve months of unprotected coitus aimed at conception''.
Approximately 5\% to 8\% of all couples remain without children unwillingly \cite{cbsStatistics, nhgStatistics}.

Luckily there are several fertility treatments.
Some of these lead to both the parents becoming biological parents. 
Others make use of donor material or surrogates, meaning that the child does not contain the genetic material of one of the `parents'.
Commonly used treatments include intrauterine insemination (IUI) and in vitro fertilisation (IVF) \cite{treatmentExplanation}.
Of these IUI is closer to regular impregnation, as sperm cells are collected and injected deep in the uterus, while IVF happens `in vitro', \ie{} in glass, outside of the body.
IVF can be further divided into more specific treatment types (\eg{} intracytoplasmic sperm injection, ICSI), 
the difference between these being the used technique or the type of paternal \silvia{parental??} materials used.
For example, there are fresh and frozen treatments; in a frozen treatment, material of the male or female is kept in (frozen) storage before being used.
Nevertheless, each treatment follows about the same steps: egg maturation stimulation, egg retrieval, fertilisation, and embryo transfer \cite{treatmentExplanation}.
The stimulation phase can also be called the start of a new \emph{cycle}. In The Netherlands (according to the \silvia{NVOG explain acronym}) 14,562 of these cycli were started in 2013 \cite{ivfReportNVOG2013}, approximately 30\% of which resulted in a ongoing pregnancy.
The success rate for a given fertility clinic is fairly well known.
However, outcome quality indicators related to the (born) child are either sparse or unknown.

All births in the Netherlands have to be entered into the perinatal registry (perinatale registratie Nederland, PRN\footnote{http://www.perinatreg.nl}).
For research purposes, however, this data is completely separated from the clinic's patient data.
The Dutch healthcare system is quite exceptional as fertility clinics are in the public domain, 
meaning that there is pressure for disclosing  data for research and governance reasons.

With minimal identifying data from both the fertility clinics and the PRN, treatment input and outcome can be linked together.
To execute this linkage the \project{} was established \silvia{[reference?]}.
During the project, data between 1999 and 2010 is gathered from each of the thirteen Dutch fertility clinics and linked to the PRN.
This data covers only ongoing pregnancies, as a child has to be born in order to link to the PRN.
In the given date range about 44,164 ongoing pregnancies were registered in the clinic data \cite{ivfReportNVOG}.
Linkage will inevitably result in a loss of a few percent where no appropriate match can be found.
But a considerate amount of pregnancies is available for research.
\silvia{where do you add a note saying that this part is in the scope of some other project - alexander?}
	\paragraph{What is big data?}
With the buzzword `big data' people often associate terms like size, volume, and analytics.
However there are a lot of data challenges which can lead to data being classified as big data.
McAfee describes it as `volume, velocity, variety' \cite{dsb1mcafee}, lots of data coming in at a high pace from many different sources.
Jacobs thinks it is a changing perspective of technical possibilities.
In the 1980s 100GB of data was considered big, now the perspective has changed, what you try to do with the data makes it big or not \cite{dsb5jacobs}.
Lynch makes it a problem of `lasting', how do we model and keep the registered (sometimes unique) events \cite{dsb3lynch}.

There are wide gaps between these definitions but also similarities.
One spanning idea about big data is that it can help understand specific domains and help make decisions \cite{dsb2lohr}.
Jacobs even states that transactions and storage of data are already largely solved problems \cite{dsb5jacobs}.
This leaves decision making, modelling, and lastingness as the main challenges.

Big data in the corporate world mostly means management and quick reaction on real life events.
Data-driven decisions are better than expert-opinion decisions \cite{dsb1mcafee}. 
A good example being flu prediction, Google is faster at predicting hospital visits related to flu than the official government sources \cite{dsb8dugas, dsb1mcafee}.

The modelling of data reflects the event in the real world which is of importance for data interpretation.
The last challenge is lastingness, \eg{} losing data can be of significance as each event is unique and will not reoccur.
There are also side effects, keeping any data (specifically medical) on persons raises privacy challenges \cite{dsb1mcafee}.

\paragraph{The data challenge}
In the context of the \project{} two of the big data factors lead to challenges, decision making and lastingness.
These are mainly human related or procedural, \eg{} ethics, trust, expectancy, lack of organisational support, etc.
Modelling of data is (currently) quite straight forward, mainly data has to be ready as input material for popular statistical software like SPSS or R.
Introducing computerised decision making may result in semantic or metadata problems, but compared to others these can be handled quite easily.

%Currently, in the case of the \project{}, decision making is currently left for researchers.
In the current workspace researchers make decisions on many levels, \eg{} what relevant hypotheses exist, which research hypothesis to pursue, what data should be analysed, how data should be interpreted.
Many of these decisions can be supported with computerised systems.
For example a hypothesis sweep can be executed with data mining operations, finding correlation in the data.
However, clinical researchers hold on to generation of hypothesis based on expertise, possibly leading to missed (important) conclusions.
This might describe a trust or expectancy issue with computerised systems, or the value of such a system was never shown.

On the other hand are the problems lastingness poses.
Funding bodies demand more of researchers considering data-management and sharing \cite{dsb3lynch}.
These demands can even extend beyond the duration of the funding, resulting in long lasting storage issues but also providing more opportunities for reuse.
For individual research projects this can be problematic as decisions on this level should be moved to institutional control \cite{dsb3lynch}.
Because assisted pregnancies are relatively rare and data gathering is a troublesome process reuse should be encouraged to make the effort useful and significant.

Lastly, lasting and reuse of data will also throw up barriers for the data deliverers.
Right now success percentages of clinics are being published as this is required by law, however they complain that the patient mix between clinics is unfair.
Clinics want to cooperate in the \project{} but they are scared that outcomes will be published in a way that will reflect directly on individual clinics.

%3 - Big data- How do your data grow?
%It also includes defining and recording appropriate metadata — such as experimental parameters and set-up — to allow for data interpretation
%This is best done when the data are captured. Indeed, descriptive metadata are often integrated within the experimental design. Description includes tracing provenance 

%1 - Big data the management revolution
%Perhaps even more important are skills in cleaning and organizing large data sets; the new kinds of data rarely come in structured formats. Visualization tools and techniques are also increasing in value.
%The best data scientists are also comfortable speaking the language of business and helping leaders reformulate their challenges in ways that big data can tackle
%The technologies are new and in some cases exotic. 
%It's too easy to mistake correlation for causation and to find misleading patterns in the data. 
%few things are more powerful for changing a decision-making culture than seeing a senior executive concede when data have disproved a hunch.
%They'll be valued not for their HiPPO-style answers but because they know what questions to ask
%When it comes to knowing which problems to tackle, of course, domain expertise remains critical
%Big data's power does not erase the need for vision or human insight.
	\paragraph{Using IT as leverage}
Summarising the challenges come down to a change of attitude.
Even though literature describes big data as a benefit for the users of it medical researchers are shying away from using it.
How can they be convinced that following certain big data guidelines can evolve performing research itself?

This paper shows a proposal for a supportive system which can manage data produced by the \project{}.
It is meant to show what value can be delivered if some human performed functions are left for a computerised system.
In order to give direction to the development the following main problems were described: security, access, browsing, querying.
This resulted in the following research questions:

\begin{enumerate}
	\item How do we implement a user-friendly system in a IVF medical domain which covers problems concerning: data security, data access, data browsing, and data querying?
	\item What needs to be changed in the current attitude towards data usage to promote big data in a IVF medical domain?
\end{enumerate}

Which were broken down into sub questions (respectively):

\begin{itemize}
	\item What are the functions of this system and which parts of the research process should this system support?
	\item Who are the users and what are the use-cases for these users?
	\item What are the legal and security aspects of this system?
	\item What is the data model for this system?
	\item What functions were actually implemented in the prototype?
	\item To what extent does this system meet the expectations of users?
\end{itemize}

And:

\begin{itemize}
	\item What are the blocking aspects of data usage?
	\item What are the promoting aspects of data usage?
	\item What alignment needs to take place to promote data usage?
	\item How can IT be leveraged to achieve this goal?
\end{itemize}
	
	\chapter{Requirement Analysis}
	\label{requirements}
	
	In this chapter the requirement discovery for the \ivfsystem{} will be described.
	At the start of the project the assumption was that the system would encompass data management (\eg{} search, select, download) and data analysis (\eg{} support of SPSS or R).
	This was however, without any knowledge about the \projectdata{} as it was not available yet.
	The dataset is not the scope of this study but a short description is necessary to understand the development process decisions.
	
	\paragraph{\projectdata{} availability}
	Data should have been available at the start of the study but it proved to be much more difficult to gather data from the different clinics.
	The major part of the problem is the necessity of bureaucracy, as medical data security lies mostly in consent procedures.
	
	In order to evaluate research protocols and data contracts ethical committees are used.
	The \project{} involved multiple sites and each of these would only allow data to be released after their own committee gave permission to do so.
	Furthermore, the evaluation processes can take quite a long time (up to one year) as some committees only meet a couple of times per year.
	
	On the other side, there are also technical issues.
	Early on in the study most of the thirteen clinics had a vendor specific electronic health record (EHR).
	Luckily during the study the adoption of a single EHR started to increase, resulting in a mostly standardised data query for a great portion of the clinics.
	One other major drawback is the fact that internet is deemed unsafe for data transfers, requiring data gatherers to travel to each of the clinics to physically pick up the data.
	
	Data is not the scope of this study, however the problems encountered with data gathering are.
	Chapter \ref{position} goes into a discussion of these problems and will propose solutions to them.
	
	\paragraph{Proceeding development}
	The named data issues counted up to a delayed delivery of the data, but also to a delay of the development discussed in this paper.
	The data gatherer of the \project{} had to be supported in technical issues as they were not equipped with the required expertise.
	Providing this support took time, but the lack of data also meant that developing the system ran into some problems.
	
	Bringing this system's concept into a brainstorm session proved to be quite difficult, without experience with the data it was too abstract for the users to form useful ideas.
	Therefore, a study was performed to find potential requirements and further define the system's description (\ie{} make it less abstract).
	This study consisted of literature studies, an interview, and observations.
	
	The literature study, resulted in descriptions of security issues and solutions.
	These are the underlying requirements of the \ivfsystem{} and have to be implemented as a result of the sensitive data repository.
	The interview tied abstract security concepts together with real-life situations.
	And lastly observations led to the concept of the work process that had to be supported. 
	
	With this input an initial requirement analysis was created which in its place was used as input for a brainstorm session.
	The results of this session were then used to update the requirements to form the final concept.
	
	\section{Security}
\label{security}

When doing research in the medical domain a well known workflow is often used. Nwogu \cite{nwogu} has formally described and defined this workflow for scientific reporting purposes (\ie{} writing scientific papers).
This workflow includes: problem definition, formulation of research question, definition of methods, data aquisition, statistical analysis, analysis results, and conclusion.

Clinical research (\eg{} a trial) is well suited to follow this workflow, as each step can be executed in turn.
Of these steps data acquisition is often the most time-consuming part.
However, acquisition of new data is not always necessary,  desirable or even possible. 
Research data of high quality and trustworthiness is valuable and should be preserved and re-used, under well controlled conditions. 
This is also the case with the \projectdata{}.

The goal of the proposed \ivfsystem{} is to facilitate reuse of this data.
There is, however, one major restriction with data sharing and re-use: medical data is (almost always) highly sensitive and must be secured. 
This imposes strong conditions for reusing the data, which have to be taken into account by the system.

The following section is the discussion portion of a security study.
Literature was searched for security issues and solutions to these issues in systems related to the clinical domain.
Found security (abstract) aspects were applied to real-life examples gathered in an interview with a software engineer working on systems supporting a big clinical registry in the Netherlands.
The full security review and interview transcript can be found in appendix \ref{security-review-appendix}.

%TODO tekst inkorten, specifiek voor het systeem maken
\paragraph{Security Analysis: the \ivfsystem{} case}
\label{security-summarisation-analysis}

Points taken from this security study will be described and reviewed here in the context of the \ivfsystem{}.

Starting with consent, for the \project{} it can be viewed from multiple perspectives: patient, clinic, and registry.
When a researcher wants to use the dataset available in the \ivfsystem{}, they will use data coming from the clinics, which in turn gather data from patients.
This patient data is then linked to the PRN registry data.
Each of the parties involved should to some extent be able to determine if they allow their data to be used.

Patient consent is a difficult problem to tackle in research in general.
When giving consent, patients need to know what they are signing for and handling data outside of the goal which was described is forbidden.
However, when using datasets for which it is unreachable and unreasonable to acquire consent from each patient in them, there are exceptions in the Dutch consent regulations.
This exception is what the \ivfsystem{} currently leans on. 
It uses historical data for the years 2000 to 2010 and according to the nationwide IVF report \cite{ivfReportNVOG} there are approximately 4000 pregnancies per year.
Which means that there are about 40.000 patients in the dataset in total.
Given the size and age of the dataset it was deemed unreasonable to require consent.
To determine if consent is not a requirement, advice from external parties should be acquired.
In this case these were: the \AMC{} chief privacy officer, medical ethical commissions of data suppliers, and the \PRN{} privacy commission.

Consent from clinics and registries can be compared to patient consent.
They all give permission to use \emph{their} data for a specific cause as described in the consent.
The main difference between these data providers in giving consent is that their considerations are based on different interests.
For example, a patient might be concerned about his/her privacy.
Of course a clinic will also take this into account when a dataset is requested but they also have interests like: what research will be performed with the data.
If this clashes with a research of their own it is less likely the clinic will give consent.
In the \ivfsystem{} these different levels of consent must be taken into account  to be able to perform the function of providing research data.

In order to fulfil regulations and ethical needs a dataset should be minimised, so that no superfluous items are left in the dataset.
For each of the data items in the dataset a purpose should be described. 
A proper purpose is leading in ethical discussions about whether to accept a data item in the dataset or not.
Having a well-defined protocol with the \ivfsystem{} can provide more confidence in the system by users, leads to better understanding of the system, and provides evidence that choices about data items were made with certain considerations.

For data linkage some identifying (\ie{} private) data items are needed.
This can be described in the purpose of the data item, but there are also methods for avoiding these data items.
Hashing of data with the application of Bloom filters \cite{something} makes it possible to link two datasets without revealing the identifying data.
On-line data linkage is only mentioned as a future work for the \ivfsystem{}.
In the first implementation, linkage is provided by a third-party and the linked data itself is seen as an external service in figure xx.

Anonymisation and pseudonymisation should be used to de-identify individuals.
While identification through data aggregation and cross-referencing is still possible to happen, these steps should make it more difficult.
The \ivfsystem{} will use both techniques to provide privacy, datasets are mostly kept clean by removing all identifying data at the data gathering step.
Whatever identifying data is left (through linkage) will be pseudonymised before it is accepted into the system.

In order to decrease the chances of cross-referencing and data breaches in general, auditing should be applied.
This means keeping logs on who uses what data at what point in time and \silvia{?? what that data looked like at that time}.
Apart from privacy this also makes it possible to keep people accountable and to provide research data management functionalities such as archival and provenance.

Lastly, exploring and using present day standard security measures are a must-have for a good system.
During the software engineering cycle of the \ivfsystem{} searches will be done for the appropriate security measures for each part of the system.
Also the expertise of developers, engineers, and system administrators with multiple years of experience each will be used.
	
	\section{Process Analysis}
\label{process-analysis}

The following section contains the aggregation of the requirements study.
It integrates the security review with the observations made at the gynaecology department at the \AMC{}.
What is described is the research process as observed by an outsider.

\begin{figure}[!h]
	\centering
	\includegraphics[width=1.0\linewidth]{images/research-workflow}
	\caption{
		This figure shows a simplification of the research workflow often used  in the medical domain (based on Nwogu \cite{nwogu}).
		The workflow components are mapped (dotted lines) by the identified \ivfsystem{} function groups.
		%Note: the data acquisition component requires execution of acquisition methods defined in the methods component. \allard{remove note?}
	}
	\label{fig:research-workflow}
\end{figure}

\paragraph{Research with the \project{} dataset}

When doing research in the medical domain a well known workflow is often used. Nwogu \cite{nwogu} has formally described and defined this workflow for scientific reporting purposes (\ie{} writing scientific papers).
The {\tt Research Workflow} in figure \ref{fig:research-workflow} shows the simplification of this workflow, which includes problem definition, formulation of research question, definition of methods, data aquisition, statistical analysis, analysis results, and drawing a conclusion.

Clinical research (\eg{} a trial) is well suited to follow this workflow, as each step can be executed in turn.
Of these steps data acquisition is often the most time-consuming part.
However, acquisition of new data is not always necessary,  desirable or even possible. 
Research data of high quality and trustworthiness is valuable and should be preserved and re-used, under well controlled conditions. 
This is also the case with the \projectdata{}.

For reuse purposes three actors are important: researcher, data manager, and interested third parties (\eg{} clinics, public, government).
Researchers are actors interested in analysis of the \projectdata{} for scientific ends.
The data manager is the central point of communication for everything that has to do with the dataset, but is also responsible for keeping an overview of everything that happens during the steps in the process.
Lastly, third parties are actors interested in research conclusions and possibly aggregated (statistical) data from the \projectdata{}.

Currently when a dataset like the \projectdata{} is exploited for reuse the following happens.
A researcher asks what data is available and can search to find what he/she needs.
Then the researcher formulates a data request and a permission granting process is initiated by the data manager.
A request contains the necessary information to base a permission decision on, \eg{} problem background, research question, perceived methods to answer question, and the requested data.
The research committee evaluates the request and based on this the researcher gets permission to receive data.

Observations and the security aspects made it clear that data requests had to be added to the system.
This differs from the initial assumption of a data management (\eg{} search, select, download) and analysis system.
Focus of the system shifted a little with the data request addition, however it is still assumed that the main interest for the users will lie at the other functions.
The acquisition and analysis processes overspan a big chunk of the research workflow, which is depicted in figure \ref{fig:research-workflow} with the gateway function groups.
	
	%TODO make paragraph

\section{Initial Concept}

\silvia{this section is too difficult to understand, need a complete rewrite}

\silvia{i think this should be a separate section. give it a name (seed concept? seed design?) - just "seed" is not enough. There was also some study done prior to this design (interviews, study, observation).
briefly report this in 2.1. Then you can present the first design itself (fig 2.1)}

Restricting the use of data to one user is undesirable, therefore good data management has to be applied to enable re-use.
\silvia{explain where the initial idea came from (maybe this is the link with the previous part?)}
The initial idea in this project was to develop a data management system with data-centric functions, \eg{} request, upload, download, search, for .....
These are represented in figure \ref{fig:research-workflow} with the blocks `request' and `data'.
With these functions a big part of the research workflow would be supported.


\silvia{
the txt below was at the caption, but i think it should not. captions are used to explain the figure - nothing else. generic stuff goes to the text body:
External services such as .. are provided ?? and outside of the scope of this paper.
Two direct users and several external users are planned each with their specific set of functions.
		Data listed is either available at initialisation of the system or is generated during execution.
}
		
\begin{figure}[t]
	\centering
	\includegraphics[width=1.0\linewidth]{images/brainstorm-before}
	\caption{
		Initial concept for the \ivfsystem{}, encompassing data and user management. 
		The system offers different sets of functions for three user roles (a,b,c) indicated by colours. 
		External services provide data according to regulations.
	}
	\label{fig:brainstorm-before}
\end{figure}

Figure \ref{fig:brainstorm-before} describes the full view of the seed. The function groups are expanded into: users, external services, data, and functions.
The external services provide ...
They influence the system but are outside of the scope of this paper.
This means that data and protocols (necessary for lawful execution of the system) 
are already provide by these external systems, and ???.
From the provided data, unlinked data is \project{} data split into clinics and PRN, where linked data are the matched rows from both.
Projected direct users are researchers and data managers.
Furthermore, external parties (clinics, public, government) might be interested in statistics of the system and its data, \eg{} aggregated and anonymised analysis outcomes.

Functions for each of the user roles encompass data management and user management.
Data functions meaning mutations of the raw \project{} data, where metadata is metadata of this raw data. \silvia{need to explain in intro what is data, metadata in general and in the context of this project}
The data block refers to data being stored in the system at initialisation and during execution.
The linked set (\ie{} raw data) is an initialisation input of the system.
Other items are generated during execution, \eg{} subsets are created after a data request is granted which also results in provenance data, etc.
	
	How is research normally performed

Workflow description

Initial idea (before brainstorm)

Idea after brainstorm

Differences between ideas

Functions in workflow

What points in the research workflow are supported/improved
	
	\begin{figure}[t]
	\centering
	\includegraphics[width=1.0\linewidth]{images/brainstorm-after}
	\caption{
		\ivfsystem{} schema after brainstorm, encompassing data, user, request, and publication management.
		The system offers different sets of functions for three user roles (researcher, data manager, and committee) indicated by colours. 
		External components are (offline) essential parts for system (\eg{} data, regulations) but are outside the scope of development (and by extension of this study).
		Data listed is either available at initialisation of the system or is generated during execution.
		*: The data dictionary contains information about all the available data items, also called: headers.
		\allard{needs to be clear that data is stored as a table, where data is stored in rows and columns, the column names are called headers here}
		**: Fields are the `raw' data that belong to a stored pregnancy, fields are named with headers.
	}
	\label{fig:brainstorm-after}
\end{figure}

\paragraph{Final concept}

After the identification of the differences that were observed during the brainstorm session the final concept can be defined.
This section describes how the initial concept was transformed to form the final concept.

The final concept in this project is to develop a system for the \projectdata{} with capability for management for: users, requests, data, and publications.
While the functions in the initial concept were educated guesses, now they are validated by the brainstorm session.
Figure \ref{fig:brainstorm-after} describes the full view of the concept, which folds back into the function groups as shown in figure \ref{fig:research-workflow-after}.

Three direct users are planned each have their own set of functions with the data they use and produce.
The (offline) external components remain nearly unchanged, only the unlinked data has been removed, the protocols are still vital for the system.

A few changes were made to the system's data.
The linked set had to be made anonymous considering the clinic, meaning that clinics would not be directly comparable against each other.
Also data about the request and publications is now stored in the system, these are both captured in a `research'.

A lot of differences exist between the schema before and after the brainstorm, an side-by-side comparison is supplied in appendix \ref{brainstorm-before-after}.
The main assumption of a data support system had the wrong focus, it is now supplemented with increased request management and the addition of user and publication management.
While data handling functions like searching, security restriction, auditing, or annotating with metadata are important in this system there are more side effects of research that have to be taken into account.

\paragraph{Conclusion: requirements}
Most of the requirements for this system are functional.
This means that the specific needs of end-users of the system lead to the requirement.
There are also a few requirements which are non-functional, these are derived from the external components and security.
All requirements are listed in the `System' section of figure \ref{fig:brainstorm-after}.

\allard{Put list of requirements in text as well?}
	
	\chapter{System Design \& Implementation}
	\label{system-functionality}
	
	\silvia{i think you can start directly with text, w/o subsection. every chapter needs an intro anyways}

\section{Functional Design}
\label{function-design}

%During the brainstorm session many new functions were discovered.

\silvia{there is too much talking about the brainstorm in this section. the requirements (from brainstorm, literature etc) should become clear in section 2, and then here you just refer to the requirements. which makes me realize that one subsection of chap 2 should be "system requirements", where you put the final set. i guess in sect 2 there are functional requirments (the function grpups) and non-functional - security, etc. Then here in sect 3 you say that the prototype will address part of the requirements (be explicit about which ones) and move on with the system design. i think fig 3.1 is the best to illustrate the flow and the functions (just make sure it is complete or put clearly in the caption which part is missing}

As described in section \ref{brainstorm} the system functions were assigned to a certain function group for management of Users, Requests, Data, and Publications.
With this knowledge the functions can be visualised with simple ordering based on the groups and prospective users as seen in appendix \ref{identified-functions}. \silvia{im not sure this "simple ordering" is relevant. the diagram should be in the body, or perhaps it is not relevant. Also, to which extent this diagram is another representation of fig 2.4? and figure 3.3? I suggest to keep one good and complete figure containing the identified functions in the body - and  it seems to me that the right place if chap 3, under "design"}
Each of the groups describe the management needs for the specific step in the research workflow (as described in section \ref{brainstorm} figure \ref{fig:workflow-after}).
When also considering the workflow the simple ordering can be fitted over it.
The result of this is shown in figure \ref{fig:functions-workflow} and will be described in the following paragraph.

\silvia{for me the text starts here}

\paragraph{Visualised}
In figure \ref{fig:functions-workflow} each of the functions belong to a certain actor (human or system), this is depicted by the colour of the actor block (light shade) and the corresponding colour of the function block (dark shade).
\allard{Think of some way to display without colours, not visible in black/white printing.} \silvia{dont worry}
The function groups are exempted from this rule as they only exist to give structure to the figure.
During the brainstorm session weight was given to the requirements, functions with less need for immediate implementation are displayed greyed-out (\ie{} change data, data curation, analyse, store outcomes).

\silvia{this is too long. the pic is clear}
When traversing the research workflow one should start from the {\tt user} block.
From here the {\tt register} function is the first step, the researcher registers as a user of the system.
After registering the account is send to the administrator user for {\tt approval}, hence the red colour for the function block.
Etcetera.

\silvia{present the story first, then the functions}
Each of the functions are mapped out according to the following story:

\begin{quotation}
	\noindent A researcher wants to investigate a certain hypothesis on the \project{} dataset.
	He or she needs to register an account with the system which is then checked and approved by the data manager.
	
	Next, the researcher formulates a data request using the system.
	From the data dictionary the researcher searches (filters) for the appropriate data items (names of data items are called ``headers'').
	The researcher creates the request with the necessary information required by the committee to decide (data headers and research question).
	The system provides feedback based on the selected fields and automatically detected keywords.
	Based on this feedback the researcher can edit the request or send it for approval.
	The committee checks and approves the request.
	
	After approval the system creates a subset of \project{} data containing the requested data items.
	The researcher filters this subset and downloads a selection of the data.
	Another possible path is that the researcher prepares the data for analysis on the system and the outcomes are stored.
	
	To complete the request the researcher uploads the, paper which is then again approved by the committee.
\end{quotation}

\begin{figure}[!htb]
	\centering
	\includegraphics[width=1.0\linewidth]{images/functions-in-workflow}
	\caption{
		Gateway functions according to function groups, actors, and usage within the research workflow.
		Vertical columns represent different function groups, colours are used for different user roles, and arrows indicate sequence of actions.
		Greyed-out functions are deemed less important, which was an outcome of the brainstorm session (see section \ref{brainstorm}).
	}
	\label{fig:functions-workflow}
\end{figure}
	
	\section{Re-use}
\label{reuse}

\subsection{Review}
\label{reuse-review}

For the reuse portion of this system the main criteria was data management.
This is the most significant part of the \ivfsystem{} and therefore should be well implemented.
Also implementation had to be done in a short time due to study planning restrictions.

Multiple systems were considered and evaluated.
Because extensions were needed the system preferably had to be open-source.
Three systems were included in an more in-depth evaluation, two externally developed open-source systems and one in-house project.
The external software was identified through the paper of Leroux \cite{leroux2011}.

\paragraph{External system evaluations}
The external systems identified were OpenClinica and DADOS Prospective.
Both are open-source and for OpenClinica a demo is available.
These software packages are called clinical trial management systems, which quickly becomes apparent when using the system.
Focus lies on data entry and retrieval for low-level (researcher) users, and on research overview for high-level (management) users.
Overview meaning displaying statistics on participants and the clinics they belong to, how many inclusions were made, follow-up percentages, etc.

Per study data collection protocols can be defined, when starting a new study the definition functionality is very flexible, .
After the protocol has been defined it is fixed for each of the study participants.
This is very useful in longer running (prospective) studies where collection should be standardised for quality, analysis, and reuse purposes.
In principle, as far as notable for OpenClinica, there are no features supporting data sharing and/or reuse.

\paragraph{In-house development}
A system named Rosemary has been developed in-house at the same department this study was performed.
Handling data, applications, submissions, and input and output of these submissions is what it was build for.
The context domain is neuroscience, the data used comes from MRI machines and are mainly references to images and their metadata.
These are then submitted for processing by applications resulting in outcomes (also stored in the system) used in further analysis.

Considering the data management part, there is data input, filtering, and reuse possibilities.
Input is restricted to automated functions and there are no manual input user interfaces available.
However, once data is in the system it allows for each data item to be used in access restricted subsets.

\paragraph{The better pick}
It quickly became apparent that no existing system would be able to cover all the needed functionality, even when looking at only data management.
The decision was made to use the in-house Rosemary project for further development.
The choice was made based on several considerations.
Flexibility in the code was an important item \ie{} open-source, extensible, easy to develop.
Next to this short lines on expertise were useful for a quick development process, once a (coding) problem was encountered it could be solved in a matter of hours.
Lastly, during the evaluation process it became clear that the data model could be reused with minimal effort.
	\section{Rosemary Design}
\label{reuse-rosemary}

The overall workings of Rosemary have been described in the previous section.
Below the architecture and data model will be explained before going into the implementation as done in the \ivfsystem{}, which will be described in the next section.

\begin{figure}[!hb]
	\centering
	\includegraphics[width=0.7\linewidth]{images/rosemary-architecture}
	\caption{
		Rosemary architecture with domain specific items removed.
		Describes the implementation of the back-end and the front-end.
		Together with the specific build tools used during development.
	}
	\label{fig:reuse-rosemary-architecture}
\end{figure}

\paragraph{Architecture}
For the Rosemary back-end the Play Framework \cite{play} is used.
It is written with the Scala \cite{scala} programming language which is interoperable with Java.
As a database mongoDB \cite{mongo} is used, a document-oriented database.
Communication between the back-end and the database is done with JSON \cite{json} and the Scala library `Salat' is used to serialise the JSON information into Scala classes.
The back-end exposes a RESTful API which can be accessed by the front-end.

The front-end is based on the AngularJS \cite{angular} framework and coded with Coffeescript \cite{coffeescript} which is compiled into JavaScript.
For layout and styling HTML and Less \cite{less} are used, Less compiles into CSS.
To provide a pleasant user experience the front-end is developed as a web application, data is loaded asynchronous through the RESTful API and stored at the client's side to give the feel of a native application.

%\begin{figure}[!hb] 
%	\centering
%	\includegraphics[width=1.0\linewidth]{images/datamodel-clean}
%	\caption{
%		Rosemary data model with domain specific items removed.
%		Describes the workspace, tagging, datum, and notification models.
%		The unedited Rosemary data model can be found in appendix \ref{unedited-datamodel}.
%	}
%	\label{fig:reuse-rosemary-dm}
%\end{figure}

\paragraph{Data model}
The yellow components in figure \ref{fig:implementation-rosemary-dm} depict the Rosemary data model with the domain specific items removed.
The model contains six main data objects: {\tt Datum}, {\tt Tag}, {\tt Rights}, {\tt User}, {\tt Notification}, and {\tt Thread}.
The {\tt Tag}, {\tt Notification}, and {\tt Rights} objects are inherited to describe a specific instance, for example {\tt WorkspaceTag} is a kind of {\tt Tag}.

A short description of the main objects will be given for better understanding of the model.
A {\tt Datum} is a single piece of data and its metadata, for the Rosemary case a {\tt Datum} might be a reference to an MRI image and metadata about the scanned patient.
{\tt Datum}s may be tagged with the {\tt Tag} object, for example, the {\tt UserTag} is a tag defined by the user and attached to a set of {\tt Datum}s for identification.
{\tt Tag}s are also used to manage access rights, each tag has a specific {\tt Rights} attached to it.
Based on the set of {\tt User}s which is assigned to the {\tt Rights} the system can restrict access to certain data.
The {\tt Notification} object stores data about process milestones which can be displayed in the user interface, for example when a message is send.
Lastly, the {\tt Thread} keeps track of messages send back and forth in conversations.

The data model and its implementation provide some interesting possibilities.
For example, a {\tt Datum} can be tracked and reused endlessly by applying a {\tt Tag} object.
One use of this is access control, which can be applied by tagging a {\tt Datum} with a {\tt WorkspaceTag} which is owned by a {\tt User}.
Many different constructs of this sort can be achieved without ever touching the structure of the actual model itself.
%The reuse in the \ivfsystem{} relies heavily on this concept, which means only slight changed had to be made to the original data objects.
	
	\section{Implementation}

% Functions as implemented
% Architecture as implemented
% Datamodel as implemented
% Design as implemented

\silvia{instead of presenting what has been done, this section is about what has not been done. it does not seem useful to spend so much space on describing things that were not done.
please make sure to describe the complete story (all requirements and all functions) in one place, and mark those that were not included in the prototype. the discussion about what was included and what was not can be part of discussion and conclusions - and not mixed in the implementation description. here you should clearly present what have been done, with technical details but with scientific language (methods, results, evaluation). }
%\paragraph{Division of management tasks}
\silvia{too long - just explain the prototype implementation here. it is not necesary to explain the dev along time, but just how the prototype became in the end}
Due to time restrictions in this study not all functions that were discovered in chapter 2 can be implemented.
A selection is made based on the programmers opinion what would be most profitable for a prototype system.
The selection that is made can be seen in figure \ref{fig:functions-implemented}.
Decisions are based on the fact that the demo has to appeal to a wide variety of users, most importantly research and clinic management.

The most basic system data management functions are implemented, namely 
search and filtering for the researcher and data manager roles, and the download function.
Data audit is implemented through placeholders to show the value of such an function without the actual data processing back-end. \silvia{what do you mean by placeholders (my spelling says this word does not exist)? is this fake functions with pre-defined responses? or a no-op?}
\silvia{??Metadata handling is not implemented, as data is considered \emph{fixed} after system initialisation.}

After this, request management is implemented.
Removing direct access to data and adding more control for committee users (coming directly from each clinic) is deemed to increase system trust by clinic management.
Request creation, editing, and submission by researcher users is supported.
Approval by committee users and automated subset creation by the system is also implemented.
Not implemented are: annotation and feedback on requests, stale request detection, and keeping provenance.
All of these functions are mostly supporting users in performing tasks but are not critical to the execution of the system.

User management is implemented with the addition of roles to user accounts and a dashboard for the data manager user.
In this dashboard the manager can find all the users of the system and assign the appropriate roles to each of them.
Lastly, publication management as closing stone of the research cycle is left out.
At this moment the additional time that has to be spend including these functions does not outweigh the added value for prospective demo users.

\paragraph{Security}
Even though security considerations are a big part of the requirement analysis (see section \ref{security}), it does not show itself that clearly in the system implementation.
Most of the security measures were taken during the data gathering steps.
Because the decision was made to have a fixed dataset for the system a lot of the discussed security measures do not need to apply anymore as described in section \ref{security-summarisation-analysis}.

Provenance, as part of security, was not implemented either.
It would be very useful to show the strength of the \ivfsystem{}, data protection through provenance can be made highly visible and easy to understand for humans.
However, the goals to quickly support the provenance functions and develop an interface on top of the data could not be reached within the time frame.

\silvia{move this to discussion}
Lastly, in order to obtain a working system a few more things are needed.
More functionality needs to be implemented, most notably better user interfaces for data filtering and selection (see section \ref{evaluation}).
Furthermore, more design and programming iterations need to be done before the request and publication management have been fully developed.
Then a big security iteration should be done to prepare the system for testing and certification.
Right now security has been patched up in the front-end by hiding information for certain users, this should change to the back-end not \emph{providing} that information to begin with.


	%\subsection{Rosemary}

\silvia{im missing system architecture, components, technologies used in the implementation. I also think that a much better explanation of the main concepts (tag, workspaces) is needed to enable understanding of the explanations below.}

\silvia{this title reads like evaluation. but the contents seems to be part of "data model" in the previous section}
\paragraph{Testing the data model's flexibility}
\begin{figure}[!b]
	\centering
	\includegraphics[width=1.0\linewidth]{images/datamodel-adapted}
	\caption{
		Rosemary data model as implemented.
		Describes workspace, tagging, datum, notification, and research models.
		The original data model can be found in figure \ref{fig:reuse-rosemary-dm}.
		Differences between the implemented model and the original model are shown in blue. \silvia{keep only one of the two figures, with color for the differences}
	}
	\label{fig:implementation-rosemary-dm}
\end{figure}

As mentioned earlier in section \ref{reuse-rosemary} alterations had to be made to the original Rosemary data model.
These can be divided into additions in the form of new objects and changes to already existing objects.

The only existing object that needs changing is the {\tt WorkspaceTag}.
This tag is used to tell the system what subsets of data exist, where each subset is made up of one or more {\tt Datum}s (\ie{} each {\tt Datum} tagged with a certain workspace tag belongs in the subset).
Each tag also has an owner and members, all the users in this set have access to the data in the workspace.

The original Rosemary implementation does not differentiate between different \emph{types} of workspaces. 
The \ivfsystem{} has a \emph{master} workspace containing \emph{all} the raw data from every clinic, it has clinic workspaces containing data specific to a clinic, and it has request workspaces containing data for a specific request.
This is reflected by adding a workspace type field to the {\tt WorkspaceTag} object.

Requests contain a limited set of data items (headers), the \emph{master} dataset contains all the available headers.
To restrict the headers a researcher can view in his or her request workspace a visibility field was added to the {\tt WorkspaceTag}.
This field contains a set of the accessible headers for a workspace, the system can then filter this data from the \emph{master} dataset before outputting the result to the user.

Next to these two (minor) changes five new data objects were added to the data model.
These are: {\tt Research}, {\tt Approval}, {\tt Data Request}, {\tt DownloadNotif}, {\tt RequestNotif}.
The two notification objects are used to determine how information is displayed in the front-end they both inherit from the {\tt Notification} object.
This means that the methods used to extract information are standardised and these two notification objects can directly be used in the system without further need for customisation.

The other three objects are related to each other, each {\tt Research} contains an {\tt Approval} object and an {\tt Data Request} object.
These related objects are used to capture data regarding the request progress.
Where the {\tt Data Request} contains the requested headers and the {\tt Datum}s these relate to in the \emph{master} dataset.
And the {\tt Approval} keeps record of the committee users that need to give permissions and what votes were already cast.
Lastly, the {\tt Research} object is used to capture information used to base a voting decision on, \eg{} research question, study description.


\paragraph{Back-end and front-end.}
\silvia{i think this would be the place for a diagram showing the components, interfaces and the technologies (maybe as notes)}
In Rosemary the back-end provides a RESTful API for communication with the front-end, therefore the back- and the front-end are completely disjoint.
Communication is done with JSON\footnote{http://json.org/}, which leaves options open for a complete rebuild of the front-end without having to touch the back-end.
For the \ivfsystem{} the existing back and front-end were both reused. \silvia{i was thinking - how will you refer to the existing and the new system? it is still rosemary (as in framework for ...) but another system. if you define a name, referencing becomes clearer and shorter}

Starting with the back-end, some structural changes were made.
Firstly, classes and methods that were used for submission and application handling were removed.
Also the data import function was removed, however this might find a place in a later prototype of the \ivfsystem{} when incremental updating of data is used.
Then, each of the objects from the data model are implemented in the back-end through a class.
For each of the new objects a class was created with the necessary methods to store and retrieve data in them.

The most notable changes to existing code were in the data handling and security classes.
Rosemary already supports basic user handling where a user creates an account and access to the system is restricted based on this.
However the system needed user roles for a fine grated control of access to the system.
To execute this control the system requires these roles to be readable and actionable (\ie{} the system can act upon a specific role).
This is reflected in the security class which now supplies this information.

Data handling had to be changed in the sense that most of the time {\tt Datum}s would be retrieved with only a small selection of the available headers.
All the read methods in the data class have been adapted to restrict access.
This makes for easy access of data from the front-end, it just provides the back-end with a request for a workspace with a certain user.
The back-end then filters data based on this and returns the restricted dataset.

Design of front-end was done on an educated guess basis.
There was no time available for a user-centred design approach where prototypes are iterated until the best design solution is achieved.
The existing layout was mostly left in place, pages related to surplus functions (submission and application handling) were removed.
For each of the new functions pages were created where necessary.

To the architecture of the front-end no changes were made.
The implementation of back-end communication and security was sufficient for a successful implementation of a prototype.


\silvia{the lack of time to implement the prototype is not well explained. i think somehwere in your thesis you need to explain the data collection to build the repository. the different formats, difficukty to access, etc. this was the reason why you did not have timein the end to do what we planned - because the data was not there, which would make the whole requirement analysis exercise too abstract and futile. this experience also comes back in your position paper.}



	
	\chapter{System Evaluation}
	\label{evaluation}
	
	\section{Case Evaluation}
\subsection{Set-up}
Describe set-up of case evaluation.
\subsection{Cases}
\paragraph{Researcher}
Evaluation with researcher user.
\paragraph{Committee}
Evaluation with committee user.
\paragraph{Administrator}
Evaluation with administrator user.
	\section{Analysis}

As with brainstorming creating a user interface is not an exact science.
For evaluation ends design heuristics are used, and the found problems can be mapped against these.
The used list is from Nielsen \cite{designHeuristics} and contains ten famous heuristics:

\begin{enumerate}
	\item Visibility of system status;
	\item Match between system and the real world;
	\item User control and freedom;
	\item Consistency and standards;
	\item Error prevention;
	\item Recognition rather than recall;
	\item Flexibility and efficiency of use;
	\item Aesthetic and minimalist design;
	\item Help users recognise, diagnose and recover from errors;
	\item Help and documentation.
\end{enumerate}

\paragraph{System hits}
Based on the system's process and the potential as a supporting factor in doing research the system got positive feedback.
One of the testers mentioned that the system as a prototype might be used for demoing purposes.
Relatively simple functions from the system can show that thought went into the workflow of data security.
Thereby fulfilling the goal of persuading data deliverers and providing more trust.

The biggest hit is the request management process.
From multiple perspectives the system can show what requests are in progress and information is available to make request `dashboards' for management purposes.
Which also brings possibilities for monitoring by data owners.
A system like this with these management functions does not exist yet.

Furthermore, the flexibility and speed of the system are plus points.
A good example of this is the search function, when used with a basic text search term the search is fuzzy.
It searches over all the available fields and returns whatever matches to it.
However, the search precision can be extended by restricting each term to a specific field (\eg{} description:pregnant).
These specific searches can then be coupled with the `and/or' keywords (\eg{} description:pregnant and keywords:embryo).
This shows the flexibility of this function, because the search is wrapped with a dedicated engine it is fast as well.
The engine used is Lucene\footnote{https://lucene.apache.org/}, and it can provide results in a matter of milliseconds.

Design-wise the data view is a perfect example for the user freedom and flexibility heuristics.
There are three ways of viewing namely: raw data, graph, aggregated.
The user may switch between these views and can pick whichever they prefer for their current task.

\paragraph{System misses}
\silvia{move most of this to discussion chapte. this is no longer evaluation but reflection}
One of the plus points, the fact that a system like this does not exists yet, is also one of the stumbling points.
There is no template to follow and as almost everything is possible users expect a lot from the system.
During the brainstorm a lot of functions came up which were not implemented because of time restrictions.
Also, no user-centred process was used, which means that most of the identified flaws in the system are simply differences in interpretation by the designer of the system.

To develop this system into a well performing demo some work is required.
Almost none of the functions shows its full potential, performing management tasks is still a manual process mostly.
In order to use this system as a demo some more `fancy bits' are needed to win over the customers.

Design-wise there are a lot of things to clean up, most of the encountered problems can be related to the system status visibility.
For example, when the user is in the data dictionary the buttons on the basket do not account for this.
Which means that two out of the three buttons are completely out of context for what the user is doing.
System visibility problems are also reflected in the fact that testers tried to `start' the search by hitting enter and not noticing that the results had already updated.

All in all the conclusion of each of the testers was that the system shows potential but needs polishing.
If time allowed it a second iteration of the system might clean up a lot of the found problems as most of them are relatively easy to fix.
	
	\chapter{Discussion \& Conclusion}
	\label{discussion}
	
	\section{Implications}

\subsection{Overview}

In the Netherlands about 5\% to 8\% of all couples remain childless due to infertility or subfertility.
There are several treatments used to assist in reproduction these are among others: IUI, IVF, and ICSI.
While it is known how many treatments result in a pregnancy it is relatively unknown what the outcomes are for children born out of these pregnancies.
For this end the \project{} was started to gather and analyse data from both the fertility clinics as the national birth registry (\ie{} \PRN{}).

Because considerable effort went into data gathering reuse of data should be supported to get the most gain out of it.
From this the idea for the \ivfsystem{} was formulated, a system supporting data management.
During the requirement analysis it appeared that there are many more aspects which lead to effective research (and thus, data reuse).
Identified requirements were separated into management groups: user, request, data, and publication.

Data management still remained the most extensive requirement group, thus for software reuse three systems were evaluated for this.
An in-house developed project was chosen as a starting point for back-end and front-end.
With minimal changes to the data model it was possible to store and use the \project{} data.
Due to time restrictions user, request, and data management functions were implemented partly.
Publication management was left unimplemented all together.

During the evaluations it became clear that for a successful demo more `fancy bits' had to be added.
Also one or two iterations of user-centred design should be applied to remove most of the fundamental confusions in the interface.

\subsection{Connecting the dots}

%How do we implement a user-friendly system in a IVF medical domain which covers problems concerning: data security, data access, data browsing, and data querying?
%What needs to be changed in the current attitude towards data usage to promote big data in a IVF medical domain?

\paragraph{Supporting administrative tasks}

During the development of the \ivfsystem{} many hurdles in the current research workflow came to light.
Normally these might be hidden to outside spectators as researchers have found workarounds.
This paragraph will discuss what the importance of supporting administrative tasks is.

Starting with one of the most important aspects for sensitive data: security.
It is known that when humans are involved errors are inevitable when time goes by.
Humans make errors, but it is possible to help them in making less errors.
Providing overviews and insights in the process they are working with can achieve prevention.

As shown in the development of the \ivfsystem{} multiple actors are involved in the data management process.
A lot of communication will run between these actors, currently this is mostly done asynchronously and sometimes even offline.
This problem gives an clear example of security through structure.
Providing the actors with a communication channel keeps the conversation encapsulated in a single system, and therefore in a single dataset.
This dataset can be leveraged to provide managers with insight in the process and can help them spot possible errors and prevent these.

Lastly, there is also inherent technical security from a well designed system.
If all actors are involved in the development and execution of the system this will lead to a better technical security as well.
In a properly executed development cycle all requirements should be identified and then analysed on whether they are important for the system.
Because specific domain knowledge is necessary when creating a system in a clinical domain, in order to identify all requirements many experts should be involved.
Each of these will input their ideas giving a better coverage on the flaws that should be expected.

A well developed (and documented!) system also leads to another aspect: understanding of underlying processes.
The system should be the embodiment of existing research processes, sometimes parts of these can be `black boxed'.
Meaning that for people outside of the `box' it is unknown what is going on and what step the process is at.
Supporting the black box processes gives the possibility to provide outsiders with cherry picked or aggregated data.

This idea can prove useful for processes like: security, restriction, approval.
Right now these are mostly hidden from data consumers but also from data deliverers.
Involving these actors will lead to understanding and ultimately to trust.
Trust can be directed towards the system but also the institution providing the system, in this study this is also the data gatherer and owner.

In that sense the system can be used as a marketing product.
Laying bare the inner workings of the institution and showing that errors might exist but also how they are resolved.
Next to data providers institutions providing grants are highly important.
Nowadays they require that their investment in data gathering should lead to high returns for the scientific community.
Mostly this will mean that data reuse is of high priority. 
Supporting a grant request with the \ivfsystem{} will show exactly how the requester plans to provide this.

Ultimately the goal of the \ivfsystem{} is to provide reuse capabilities for datasets.
Unfortunately in research (especially in the clinical domain) this is not something that comes naturally.
Data has value, a lot of effort goes into gathering and upkeep.
Researchers and institutions want to hold on to their data or at least gain something from reuse.

When making data open access it is hard to check whether if the right references are used in publications.
On the other hand data reuse can lead to better research, for example publications can be checked by an independent institute.
With a \ivfsystem{} data can be restricted while the line of communication between data owner and consumer is short.
This will bring all the benefits of data reuse while the data owner still has a grip on what happens with it.

\paragraph{Extrapolate to other domains}
%How can this specific system be used in other settings/knowledge domains?

The previous section argued the benefits of a \ivfsystem{}, however it is specific to the domain of \IVF{} and \PRN{} data.
There are other projects building towards something similar, but no complete system exists yet.
Most follow a stepwise implementation of blocks, introducing user, data, request, and publication management one by one.
Leaving each domain to their own experts however will create a fragmentation of many different implementations of (in principal) the same functionality.

It is a good idea to work towards a modular solution which implements core elements of each of the management groups.
Different domains can pick what they need and add their specific `sugar' while the functionality remains largely the same.
This will require an investment into research on how to make the system's modules abstract enough for reuse in other domains.
And even when abstract modules exist process and requirement analysis for each implementation project will take a considerable amount of time.

Before the \ivfsystem{} is suitable for this type of reuse many hurdles are yet to be taken.
These lie both in social fields as in regulation.
The social aspects relate to building trust, managing expectancy, etc.
Regulations in the Netherlands for sensitive data are strict (as to be expected).
The \IVF{} and \PRN{} dataset is relatively light on sensitive data and skirts around some of the laws.
When more interest is shown in the \ivfsystem{} these are some of the interesting areas for further research.

Furthermore, the \ivfsystem{} is not ready for demoing purposes yet.
The evaluation of the prototype shows there are areas that need improvement.
Some are time consuming and are not directly visible, like integrating data provenance.
Others will take less time and are highly visible, like updating the user interface.
Once more, this study answers the question of what is needed to implement a successful system but does not deliver it.

\paragraph{Blocking factors}
\allard{TODO once position paper is done}
%What are the biggest blocking factors in doing research with sensitive data and how can these be overcome?




	
	\section{Critical Appraisal}

\paragraph{Strengths and limitations}
This section will be split into two parts, as there are strengths and limitations considering the system but also the implementation itself.

Starting with the implementation, the biggest limitation of the study is the limited time allowed for development due to struggles with data gathering.
This resulted in many design shortcomings in the front-end and minimal implementation of security features.
Features that are not directly visible in the front-end are not critical for a successful prototype.
However, a good user experience with the system should be reached before pitching all the systems' benefits.

Even though the thesis contains a rather extensive research into security, most of the found solutions were not used in the \ivfsystem{} implementation yet.
Even now, after the first prototype is finished, it would not be advisable to start introducing these to the system.
First another iteration of function and design development should be ran before it becomes feasible to use intricate security constructs.
For the purposes of a prototype this is not a big problem, as long as security issues and solutions are kept in mind before starting to use the system in a live environment.

While the system's prototype might be lacking user-centred design and functionality, it already illustrates the potential what it could ave after some more development effort.
In the evaluation users noted that the ideas of research workflow management can be used to attract funding because they support so well data reuse.
Furthermore, big data literature states that the value of an investment into data gathering is increased when sharing and reuse capabilities are added (see chapter \ref{introduction}).

When considering the system itself, and how it is currently placed in one specific domain, it is hard to tell how it will extrapolate to other domains.
Because the data model of the implementation is very flexible, this should not lead to problems.
In the set-up is currently the dataset is fixed after system initialisation.
This implicates at least two things: there is no data input or edits, and consent for data use is handled by the research committee only.
Data editing and input are functional requirements which can be easily implemented.
Consent and lawful procedures are quite  challenging and might change between domains.
However, as discussed before data gathering and data reuse could be separated into two individual systems, 
where the \ivfsystem{} handles the reuse once a fixed dataset is reached by the gathering process.

One last limitation of the system is that for reuse only open-source software fitting the requirements were considered.
This is also due to time restrictions.
The drawbacks here are that well implemented existing systems with functions like data gathering or data export were not considered.
Using these as building blocks in a larger cooperating framework might be more advantageous.
This would also mean that a gateway has to be further developed to encompass all the different systems into one interface.

\paragraph{Future research and development considerations}
Future research mostly refers to further development considerations
The \ivfsystem{} is only a prototype and not ready for real life usage.
Evaluation of the prototype shows that there is room for much improvement.
Some are time consuming and are not directly visible, like integrating data provenance.
Others will take less time and are highly visible, like changing the user interface.
This study answers the question of what is needed to implement a successful system, but not all solutions could be integrated into the prototype.

%During the brainstorm a lot of functions came up which were not implemented because of time restrictions.
One of the plus points, the fact that a system like this does not exists yet, is also one of the stumbling points.
There is no template to follow and, as almost everything is possible, users expect a lot from the system.
Most notably, better user interfaces for data filtering and selection should be implemented (chapter \ref{evaluation}).
No user-centred process was used, which means that most of the identified flaws in the system are simply differences in interpretation by the developer.

Furthermore, more development is needed to complete the implementation of missing functionality.
Then a significant security revision should take place to prepare the system for testing and certification.
Right now security has been patched up in the front-end by hiding information for certain users. 
But this should change into solutions in the back-end to prevent information provision to begin with.

The \ivfsystem{} could be also further developed and used in further data management research in the clinical domain.
In this domain a vast network of interweaving interests, guidelines, and laws exists.
Untangling this network should be the aim of further research into this area.
\begin{itemize}
	\item How can clinician be persuaded to provide patient data for research?
	\item How should data be structured to maximise research benefits while making sure privacy is kept?
\end{itemize}

Furthermore, research into further development of the \ivfsystem{} should be performed.
Examples of research questions are:

\begin{itemize}
	\item What type of technology can be used to improve data reuse considering patient consent?
	\item How can the system provide better support to users while they perform data management tasks?
\end{itemize}
	
	\section{Conclusion}

The focus of the developed system changed during this study.
Initially the system would be supporting researchers in doing research which was the basis for the first research question:
How do we implement a user-friendly system in a \IVF{}--\PRN{} medical domain which covers problems concerning: data security, data access, data browsing, and data querying?
The conclusion to this question will be structured through its sub-questions.

While the requirements have changed, there are still aspects of the system which attribute to the problems we wanted to address.
Required functions which should be supported and the prospective users have been identified.
These are listed in section \ref{brainstorm} in figure \ref{fig:brainstorm-after} and section \ref{functional-design} in figure \ref{fig:functions-workflow}.
Researchers, the data manager, and committee members need to be supported in functions belonging to user, data, request, and publication management.
Most notable is that the focus of the system now lies on request management and the committee members.

Of course, because the \projectdata{} contains sensitive medical data, security is an important aspect of the system.
An extensive security review, both in literature as with an interview, lists the possible issues and proposes solutions to them.
This review can be found in appendix \ref{security-appendix} and a summary is described in section \ref{security}.
Security solutions are mostly used during the fertility data gathering process, removing most of the privacy sensitive data.
No solutions were implemented in the system due to time restrictions.

To store the data in the system the data model of an existing system was reused.
This is the in-house project Rosemary, originally   developed for the neuroscience domain.
With minimal changes the data model could fit to the \projectdata{}, which is described in section \ref{implementation} and shown in figure \ref{fig:implementation-rosemary-dm}.

Functions which would be implemented into the \ivfprototype{} were decided upon with educated guesses.
The basic functions like data access, browsing, and querying were mostly covered through the reuse of Rosemary.
Most extensive development went into request management and supporting the committee users.
This prototype is not completed yet.
More functionality should be added, but most important the user interface should be redesigned.
This could, for example, be done with a user-centred design process.
However, the prototype shows what a completed system is capable of which is reflected in the evaluation in chapter \ref{evaluation}.

The second research question considers the underlying processes of a system like the \ivfsystem{}.
It tries to tie together the \projectdata{} with the concepts of big data with the research question: What needs to be changed in the current attitude towards data usage to promote big data in a \IVF{}--\PRN{} medical domain?
The conclusion to this question will be structured through its sub-questions.

Big data literature is mostly about possibilities and positive aspects.
Aspects related to the system's dataset are described in `What is big data?' and `Big data for \project{} research' in chapter \ref{introduction}.
Most notable are preservation, reuse, and analysis or decision making.
The literature describes that funding institutions push for effort into data preservation.
These efforts should lead to more data reuse even after the end of the funding grant.

The strength of big data lies with the possibilities of decision making.
Computers are better capable to analyse data and find correlations within them.
In the case of the \project{} this could be used to determine what possible hypothesis can be answered with the data.
This idea of data analysis was captured in the system's initial concept (section \ref{process-analysis} figure \ref{fig:brainstorm-before}).
During the brainstorm session these functions were considered `nice to have', but not essential.
Hypothesis are formulated by researchers and then tested on the \projectdata{}.
To support this,  preservation and reuse aspects are important.
Data should be accessible for researchers that want to test a certain hypothesis.
However, the sensitive nature of the data and the associated regulations make this difficult.
This is shown in the consent discussion in the security review (section \ref{security}).
But also in the fact that committee members have become  important users of the system.

Preservation and reuse are addressed in the \project{} project with the support of \ivfsystem{}.
We discussed that through transparency and giving back control over data the owners are more willing to expose their data for research ends.
Analysis is not currently in the system and therefore this question remains open.
We expect that the system in the future can overcome some (if not all) of the identified blocking aspects.
However, further research where the system is used in production should show this.
	
	\chapter{Data Management Position}
	\label{position}
	
	\section{Introduction}
\paragraph{Problem}
Describe data `usage' phases and what the problems are now:
\begin{itemize}
	\item Study preparation, data usage approval
	\item Data retrieval
	\item Data massaging/preparation
	\item Data analysis
	\item Data reuse
\end{itemize}
\paragraph{Position}
There should be more data re-use and cooperation between data producers/consumers.
IT can help with the alignment of these two factions.


%the lack of time to implement the prototype is not well explained. 
%i think somehwere in your thesis you need to explain the data collection to build the repository. 
%the different formats, difficulty to access, etc. 
%this was the reason why you did not have timein the end to do what we planned - because the data was not there, which would make the whole requirement analysis exercise too abstract and futile.
%no data
%difficult to gather, different formats, difficult to access
%social problems, regulation problems, technical problems (in that order)
%first the heads of clinics need to be in line, then the clinicians
%then everything needs to be aligned with Dutch regulations (it is actually intertwined with the previous one as good protocols will lead to better understanding and more likelihood that they will agree).
%technical problems, once everyone agrees we should do this and it is allowed we run into the next problem, it is hard to gather the data
	
	\section{Background and Evidence}
How is research performed now?
Describe Dutch law on privacy and data usage.
Explain how METC and commissions influence data usage.
Talk about clinicians.
Patient willingness to share their data for research.
Big data and its general stance on data usage.
	
	\section{Discussion}
(Limitations):
\begin{itemize}
	\item How does Dutch law prescribe security and ethics to be taken into consideration and why does this matter, what is the fear?
	\item Is law taken into account in the right manner every time?
	\item How do METC, commissions, and other influential bodies limit the use of data?
	\item What are the clinician's opinion, thoughts, and `holding back' in fear of rebuttal
	\item What are expected hold-ups for research now (data gathering can take extremely long)?
	\item Will patients withhold their data for use in research and why?
\end{itemize}
(Chances):
\begin{itemize}
	\item What does existing big data research describe is necessary in order to conduct successful and meaningful studies?
	\item What are some necessities for a certain fluidity in using sensitive data for research?
	\item How to can limitations be aligned with the big data opportunities
	\item What needs to be changed in attitude in order to open up data reuse and sharing (introduce Scandinavian ideas)?
	\item How can online and offline (IT) be leveraged to bring convincing and easy-to-grasp ideas to inspire people to change their way of working?
\end{itemize}
Give both sides of the issue and discuss how these two can be aligned with each other to meet at a sweet spot for both factions.
	
	\section{Conclusion}
\paragraph{Further steps}
Figure out if IT can be leveraged to align medical and technical personnel.
\paragraph{Proposing solutions}
IT leverage in the form of a system (like the one in this thesis).
	
	\clearpage
	
%	\printbibliography
	
	\appendix
	\chapter{Abbreviations}
	\label{abbreviations}
	
	\paragraph{NVOG}
Nederlandse Vereniging voor Obstetrie en Gynaecologie (Dutch association of obstetrics and gynaecology), institution looking after the interests of the Dutch obstetrics and gynaecology domain.

\paragraph{IUI}
Intrauterine Insemination, an assisted reproduction technique.

\paragraph{ICSI}
intracytoplasmic sperm injection, an assisted reproduction technique.

\paragraph{\IVF{}} 
In Vitro Fertilisation, an assisted reproduction technique.
In this paper also used to denote the medical domain of fertilisation.

\paragraph{\PRN{}}
Perinatale Registratie Nederland (Dutch national perinatal registration), registration containing all perinatal data from the Netherlands.
The registry exists of population based data on pregnancies, provided care, deliveries and (re)admissions of newborns.

\paragraph{\project{}} 
Dutch Assisted Reproductive Technology Study, a encompassing body covering multiple studies concerning assisted reproductive technologies.
In this paper used to denote the study concentrating on linking data from fertility clinics with \PRN{} data.

\paragraph{\projectdata{}}
\project{} dataset, dataset containing the linked data from fertility clinics and the \PRN{}.

\paragraph{\ivfsystem{}}
\project{} gateway, project name for the development of the system described in this paper.

\paragraph{SPSS, SAS, and R}
Popular statistical software packages.

\paragraph{EHR}
Electronic Health Record, software package collecting health information on patients.
Replaces the paper health record.

\paragraph{AMC}
Academic Medical Centre, university hospital in Amsterdam.
Institution where this study was conducted at.

\paragraph{NSG}
NeuroScience Gateway, web application providing an user friendly interface for managing data, processings, and community.
Stores image references and metadata on MRI data.
Provides support for processing this data with specific applications.

\paragraph{MRI}
Magnetic Resonance Imaging, technique used for scanning (parts of) the human body.

\paragraph{\ivfprototype{}}
\project{} prototype, name for the implemented prototype of the \ivfsystem{}.

\paragraph{CTM}
Clinical Trial Management, refers to a type of software used to gather and store data when conducting a clinical trial.

\paragraph{XNAT}
Data storage service specifically designed for storing MRI images and metadata.
	
	\chapter{Brainstorm Schemas}
	\label{brainstorm-before-after}
	
	\begin{sidewaysfigure}
    \centering
	\includegraphics[width=1.0\linewidth]{images/brainstorm-before-and-after}
    \caption{
    	Side by side comparison of system functionalities before and after brainstorm as described in \ref{requirements}.
   	}
    \label{fig:brainstorm-before-and-after}
\end{sidewaysfigure}
	
	\chapter{Security Review}
	\label{security-review-appendix}
	
	The question we seek to answer in this review is what regulations and conventions should be taken into account when storing and processing data specific to the \ivfsystem{}?
We could start with answering the question `how do I secure data?', but before doing this it is important to describe what the incentives are for using the data.
We do this to give context to the found security issues and proposed solutions for them.

\paragraph{Medical big data ethics and security}
\label{security-ethics}

In 2013 Groves \cite{s20Groves2013} estimated big data strategies could increase profits in US healthcare by \$100 billion \cite{s13Patil2014}.
However to make this possible healthcare management has to change their business model.
Old models worked with increasing or decreasing the amount of patients while the outcome (quality) remained the same; Patil \cite{s13Patil2014} calls these ``volume-based business models''.
To make the increase in profits possible, new models revolving around cost versus quality are used, called ``value-based business models''.
Much more is written about the pros and cons of this type of healthcare management by gurus like M.E. Porter \cite{s21Porter2006}, but this is outside the scope of this paper.

In order to measure the quality indicators in the new strategies big data can be applied \cite{s6West2009}. 
Through data mining, patient outcomes can be connected with treatment, environmental factors, or any other information.
When this is done right choosing the best treatment for any given new patient will be a matter of checking the charts.
This way of delivering healthcare is also supported by the European Union which describes it as ``Free Movement'' for patients between institutions, obtaining quality, and efficient healthcare \cite{s8FernandezAleman2013}.

The new set-up closely fits to a research set-up, in the sense that patient data is used for improving the healthcare process.
Data in patient records is mainly used to provide information for caretakers, but it can be very valuable in research as well \cite{s15Fenz2014}.
Because the current model of healthcare data is not fully compatible to the ``value-based business models'' extraction of data is not very straightforward.
Also, caution should be kept as there are security and privacy pitfalls.

Even though these problems exist, according to Fenz \cite{s15Fenz2014} patients will not withhold their data for research purposes.
However, there are concerns that data can be accessed by persons with unwanted intentions: marketing, insurance, or data loss through breaches.
It is a good thing that big data can help the improvement of healthcare \emph{and} patients are willing to give their data towards this end.
But data breaches should never be dismissed, which makes the application of big data an ethical question: Does the benefit of the big data overcome the possible negative effects of a data breach?

Kluge \cite{s7Kluge2007} looked into this ethical question by taking a context: ``What should be the driver of its [the systems'] development and implementation?''. 
And splitting the main question into four sub-questions, respectively: Should it be the technology itself? Should it be the interests of service providers? Should it be the interests of governments? Should it be the interests of patients?
Each of these questions are applicable to the \project{}:

\begin{itemize}
	\item \textbf{Should it be the technology itself? -} Because this system is developed in the interest of research, the technology has a stake in the driver of development.
	However, it should not become the main driver as a fully optimised system does not necessarily translate into improved quality and efficiency of care.
	Which, in an ethical sense, should always be the main driver as patient `donate' their data for this purpose.
	\item \textbf{Should it be the interests of service providers? -} If the \ivfsystem{} is implemented in a broad scope it should also include feedback.
	This is in the interest of the service providers (\ie{} clinics) as they will finally be able to get a impartial comparison between each other.
	If the feedback to providers is delivered in the right manner this should provide both the incentive \emph{and} the knowledge to improve efficiency.
	However, in the current scope of the \project{}, feedback is not accounted for.
	\item \textbf{Should it be the interests of governments? -} Because the \project{} is the first to look into the coupling of outcome data with \IVF{} data the holy grail to be reached is that the system provides the same services as registries like the NICE\footnote{https://www.stichting-nice.nl/} or BHN\footnote{https://www.bhn-registratie.nl/}.
	Providing reports to the government makes it possible to make informed guidance and policy decisions.
	\item \textbf{Should it be the interests of patients? -} The main outcome of the \project{} is to provide insight in differences in outcome between different \IVF{} treatments.
	A study with these outcome measurements and this size has not been performed yet in the Netherlands.
	The outcomes of a study will be able to determine what the best treatment is, and to conduct a study the \ivfsystem{} will support the researchers.
\end{itemize}

There are some ethical issues that also involve legal aspects like the USA Patriot Act, this will be touched upon in the paragraph \emph{the legal side of security}.

\paragraph{The `how?' and `why?' of security}
\label{security-how-why}

When speaking of security in the medical sector the topic can be discerned into several fields according to Perakslis \cite{s2Perakslis2014}: data loss, monetary theft, attacks on medical devices, and attack on infrastructure.
Because the main scope of security in this study is clinical research data, and thus protecting patient privacy, we will focus on data loss.
The term `data loss' is used in a broad sense here, loss refers to all manners of breaches where data can be accessed by unauthorised persons.
To prevent these losses data security is applied.

The use of technology is increasing in the healthcare sector and this results in systems with increased complexity, diversity, and timeliness \cite{s13Patil2014}.
As these systems grow in both size and complexity it becomes harder and harder to prevent data breaches.
Furthermore, recently (2014) it has been shown that the healthcare domain is being heavily targeted: 94\% of all institutions dealt with attacks on their systems \cite{s2Perakslis2014}.

It is reported that there is more risk of insider attackers, both intentional and unintentional (\eg{} regular employees not following policies) \cite{s1Zamosky2014}.
The most common breaches are unauthorised data access (63\%) and exposure (\ie{} being available to unauthorised persons) of sensitive data (57\%) \cite{s18Kum2014}.
Only 7\% of all breaches are caused by hacking while mundane errors like stolen laptops are more common \cite{s1Zamosky2014}.

About 48\% of data breaches can be traced back to theft, with identity theft being the most important one \cite{s1Zamosky2014}.
An obvious factor in this is that the value of stolen data is much higher than in other domains.
The per-record value is directly bound to the amount of data in the record; typically medical records contain more information than for example financial records which makes them more valuable \cite{s1Zamosky2014}.

It has been estimated that the cost for the institution (for legal actions, recovery, etc.), after confidentiality has been breached, is approximately \$233 per record in healthcare.
While the mean of all industries is \$136 \cite{s2Perakslis2014}.
Some other incentives for hackers to crack a system are to deface an institution or to show that security is lacking.

There are three goals that work towards achieving a secure system: \emph{confidentiality}, \emph{integrity}, and \emph{availability} \cite{s8FernandezAleman2013}.
\emph{Confidentiality} refers to the rights of the patient concerning their personal medical data, which will be described in the next paragraph.
\emph{Integrity} refers to completeness and correctness of data, and \emph{availability} is reached when the correct person can access the correct data at the correct time (\eg{} clinician accesses a patients' file during a consult).

As the \ivfsystem{} does not influence patient care directly \emph{availability} is of less importance.
The importance of \emph{confidentiality} however, is reflected in the following quote: ``They [clinicians] wanted to help achieve these benefits but also wanted to be sure that patients' rights were protected and that clinicians were not in danger of breaking patient confidentiality and the law'' taken from Sanderson \cite{s5Sanderson2004}.
Clinicians see the benefits of systems supporting them in their daily work, but they need to be sure that these systems are safe to work with.
Furthermore, Layman \cite{s4Layman2008} states that the chance of achieving success with a health informatics system decreases when confidentiality is violated.
\emph{Integrity} is of importance in research as the correctness and completeness of the data directly mirror the quality of the dataset.
Thus, \emph{integrity} of the data directly influences the quality and value of the conclusions drawn from it.

\paragraph{The legal side of security}
\label{security-legal}

While systems grow and hackers become more eager in hacking them, lawmakers have put regulations and laws in place to protect patients.
IT is agile while regulations (unfortunately) are mostly slow-moving \cite{s20Groves2013}, which makes it difficult for the regulations to follow the security standards of the present day.
Therefore, it should be noted that compliance to these regulations will not necessarily be sufficient for good security \cite{s20Groves2013}.
Thus, institutions that do not look for further security measurements are likely in danger of data breaches.
This paragraph will describe what regulations are in place. 
Only major points will be discussed as an in-depth explanation is outside the scope of this research.

Healthcare institutions have to comply to regulations like the European Union (EU) or United States (US) privacy directive.
Respectively the EU directive is 95/46/EG and the US directive Health Insurance Portability and Accountability Act (HIPAA).
If institutions do not comply, there are acts in place that allow government bodies to impose fines.
These fines are becoming larger as the importance of protection of privacy is growing \cite{s1Zamosky2014}.
For Dutch institutions only the EU directives apply but the HIPAA also gives a good insight into the important topics concerning privacy protection.

The EU and US directives are comparable to each other which makes it easier to find technical privacy solutions (described in \ref{security-summarisation}).
In the Netherlands the EU directives are implemented and supplemented with other acts.
Mouw \cite{s19Mouw2012} describes what the implementation means in real life. 
The  parts of the acts applicable to the \ivfsystem{} are described below:

\begin{itemize}
	\item Data Protection Act (WBP) - Wet Bescherming Persoonsgegevens (WBP), this is the implementation of the EU directive.
	This act describes when and how data can be collected from a person.
	The most important point to take from this regulation is that specific consent is required before data can be collected and processed.
	Furthermore, a government body is granted permission to enforce privacy regulations and impose fines when these are breached.
	\item Medical Contract Bill (WGBO) - Wet Geneeskundige Behandelingsovereenkomst (WGBO), this is an overall act for medical treatment that also describes regulations for record keeping.
	The most important regulation here is article 458: statistical or medical research in the benefit of public health are exempted from consent regulations if the consent required is impossible or unreasonable to obtain.
	\item Medical Research Involving Human Subjects Act (WMO) - Wet Medisch-wetenschappelijk Onderzoek met mensen (WMO), this regulation is an extension to the WGBO and requires an ethical committee's approval for each new medical research.
	\item Code of Conduct for Medical Research (FMWV) - This code of conduct provides concrete examples and implementations of each law mentioned above.
	This makes it easier for medical researchers to comply to each of these laws.
	The code requires that personal data is only accessible by the researcher or those directly under their authority.
\end{itemize}

There is one US act that is also of interest for institutions world-wide, namely the US-based Patriot Act.
The Patriot Act is mentioned here because it endangers patient privacy, even for countries outside of the US.
If a third-party is used for data storage, and this third-party has its head quarter in the US, government bodies from the US can request (and will most likely receive) insight into this data \cite{s7Kluge2007}.
This issue should be considered while developing a distributed system that handles patient data.

%Need to find those standards and examine them
Next to legal items there are also some standards that can be applied, namely ISO EN13606 and the NEN7510.
For example the EN13606 defines confidentiality as: ``process that ensures that information is accessible only to those authorised to have access to it'' \cite{s8FernandezAleman2013}.

Summarising, the laws state that in principle patient data cannot be stored or processed.
However if there is a public interest, like public health, some data may be used \cite{s19Mouw2012}.
In any case personal (\ie{} identifying) data should always be separated from the rest of the data.
Types of data such as religion, race, and sexual preference are excluded and may only be stored when the law explicitly permits it \cite{s19Mouw2012}.
When there is a public interest, data storage and processing is possible when each included patient gives his/her explicit consent.
When consent is asked, a clear description of the goals for the data should be given.
Also, the patient remains the right to inspection and removal of his/her data from the dataset.
An exception to the consent rule is made when requests are impossible or unreasonable to make.
Furthermore, the law states that medical research can only be done when : ``(I) new scientific insights in medicine are expected, (II) those insights cannot be gained in another, less risky way, and (III) the risks for and interest of the participants is well balanced against the importance of the research.'' \cite{s19Mouw2012}.

The FMWV code of conduct describes the following points which are applicable to this research:

\begin{itemize}
	\item For personal data, the reason of use should be described.
	\item The persons authorised to view personal data should be listed.
	\item Provisions taken to protect the data should be listed.
\end{itemize}

Lastly, an ethical committee is needed to check the usage of data versus the goal of the research.
If it is found that the means do not serve the goal the research cannot continue.

The intent of these laws can be captured in a great quote from Aleman \cite{s8FernandezAleman2013}: "the claim of individuals, groups, or institutions to determine for themselves when, how, and to what extent information about them is communicated to others".
	\section{Interview}
\label{security-interviews}

Next to the literature search a semi-structured interview was held with a system security and medical registration system expert to provide examples for the laws and regulations which are quite abstract.
This interview technique leaves the interviewee free to roam to other subjects which might prove useful in the design of the \ivfsystem{}.
During the interview firstly the vision of the \ivfsystem{} was explained in a few sentences.
The following questions were used to give structure to the interview:

\begin{itemize}
  \item What can we do to make sure we are not processing personal data?
  \item What are the criteria for something to be personal data or not?
  \item If personal data is being processed how can we comply to the extensive rules?
  \item How are things handled in existing registries?
  \item Are these registries also used for research?
  \item Aggregate data in public databases can become individually identifiable when the databases are integrated or the data are cross referenced. 
  How do you account for this?
  \item Facing problems with getting go-ahead from ethical commissions for gathering data, how can a system like the \ivfsystem{} support in getting consent?
\end{itemize}

In the following section the term \emph{personal data} refers to identifying data which can lead back to one individual.


\paragraph{Interpreted transcript.}
\label{security-interview-transcript}


The interview was with an expert on the topic of an intensive care registry in the Netherlands: National Intensive Care Evaluation (NICE), 
which contains sensitive data for which consent would normally be required. 
However, patients at Intensity Care (IC) are generally non-responsive, which means that getting consent is an unreasonable requirement and can be disregarded.

Considering consent, there is a difference between historical data and \emph{active} data.
When using historical data it is difficult to get consent from each patient, which provides more room for the interpretation of the law.
For active data it is fairly easy to provide patients with a consent form when they visit the healthcare provider, thereby binding the researchers to acquire consent. 

It is difficult to avoid processing personal data, \ie{} what is defined by `personal data' is open to interpretation.
Therefore, each of the used data items that might be debatable should be supported by a goal (\ie{} purpose of data use).
Goals may vary but most of the time they describe why a certain data item is inevitable to use when doing research with the dataset.

A good guideline when creating a dataset is to take the minimum amount of data items while still being able to fulfil the research goal.
Some categories of data weight more in a decision than others.
For example, \emph{sensitive} items (\eg{} race, sexual preference) are more likely to be turned down.
Once more, for ethical commissions the purpose of data collection and processing is a leading factor in a decision. 
A well described protocol and the application of standards (\eg{} NEN7510) are other factors.

It is impossible to guarantee that privacy is kept at all times.
This is due to factors like public datasets, news, and all other sorts of information sources.
When aggregating these datasets into one big dataset it becomes easier to identify individuals. 
For example, the Dutch queen is hospitalised and this information is published in a newspaper, from other sources (\eg{} Wikipedia) age and gender can be gathered.
With this information, even though the NICE registry is considered anonymous, the subset of possible patients could hypothetically be reduced until only one patient remains that could compare to the queen.

In order to avoid these kinds of data breaches some precautions can be taken, but these will never be completely safe as there is a human factor.
Precautions that should be considered are: take notice of, and implement, modern technical security techniques; keep external access to data either off-line or require to go through an internal administrator (\eg{} data extraction requests); aggregate data that is communicated to external sources which removes the likelihood of an individual being identified; when direct access to data is needed (\eg{} administrators, data mining research) use confidentiality documents; make direct access to data bound by location (\eg{} on one specific computer or inside a network).

As a side note, two topics came up while talking about trust that a people have in a system:  accountability and integrity.

The NICE registry makes it possible for ICs to benchmark themselves against data of other clinics.
However, all clinics remain anonymous, which avoids discussion about \emph{accountability}. 
For example, it is highly likely that the worst ranked IC will receive less income once the ranking is known to others (less patients willing to go there, insurance is unwilling to pay, etc.).

The second topic, \emph{integrity}, refers to objectiveness of outcome measures.
Quality indicators that are presented by the system should never be directly influenced by humans as this introduces data playing problems. 
When this happens clinics artificially try to improve their scores by giving patients better outcomes than they actually have.
	\subsection{Technical \& Procedural Cornerstones of Security}
\label{security-summarisation}

The outcomes of the interviews were combined with the literature study to compile a practical list with security issues and solutions that are relevant in this project.
The list is described in table \ref{tab:security-list}, ordered by type and paper.

Furthermore, two checklists were found in the literature.
These lists give a number of points which a system should comply to in order to cover all identified security areas.
The first list describes items used to test the safety of an implementation of a patient-centred eHealth solution \cite{s17Dehling2014}.
This check list (Checklist A) can be found in appendix \ref{security-checklists-a}.

The second list describes questions used to test EHR systems on security and privacy \cite{s8FernandezAleman2013}.
This check list (Checklist B) can be found in appendix \ref{security-checklists-b}.

% Security table with a list of all found problems and solutions in literature and interviews
\begin{center}
	\begin{longtabu}{c X}
		\caption{List of identified risks and solutions, sorted according to type} \label{tab:security-list} \\
		\hline
			\multicolumn{1}{c}{\textbf{Ref.*}}	&	\textbf{Description} \\
		\hline
			\multicolumn{2}{c}{Procedural problems \& risks} \\
		\hline
			\cite{s4Layman2008}, \ref{security-interviews} &	Data aggregation and cross referencing makes the identification of individuals possible. \\
			\cite{s18Kum2014}	&	Applying secondary data analysis makes it difficult to cover purpose in the consent process. \\
			\cite{s18Kum2014}	&	Data linkage without identifying data is impossible, data linkage with identifying data is unsafe. \\
			\cite{s18Kum2014}	&	In data linkage, even if the linked dataset has no need of person identification, privacy has to be temporarily breached in order to identify matching entities in two datasets. \\
			\cite{s18Kum2014}	&	Finding the right amount of identifying data for linkage is a difficult task. \\
			\ref{security-interviews}	&	Determining what identifying data is, is open to interpretation. \\
		\\ %whitespace
			\multicolumn{2}{c}{Procedural solutions} \\
		\hline
			\cite{s3Herveg2014}	&	Personal data can only be used for the purposes described when the consent was given by the subject. \\
			\cite{s3Herveg2014, s6West2009, s18Kum2014}, \ref{security-interviews} &	Data being used should be in a minimum dataset, no superfluous data should be present. \\
			\cite{s3Herveg2014, s15Fenz2014}	&	After the purpose described in the consent has been reached, identifying data should be removed from the dataset. \\
			\cite{s3Herveg2014}	&	The purpose of data usage should be: ``specified, explicit, and legitimate''. \\
			\cite{s8FernandezAleman2013}	&	At any point in time an audit of data should be kept. \\
			\cite{s8FernandezAleman2013}	&	Accountability is a central part of security. \\
			\cite{s15Fenz2014, s13Patil2014}	&	Apply anonymisation and pseudonymisation to protect identifiable data.
			Or make it impossible to use this data to identify individuals while still being able to use this data for analytical purposes. \\
			\ref{security-interviews}	&	For each sensitive data item describe the purpose it fulfils. \\
		\\ %whitespace
			\multicolumn{2}{c}{Technical solutions} \\
		\hline
			\cite{s6West2009}	&	Use hashes of identifying data to refer to individuals, which is usable for analytical purposes but not for identifying individuals in the real world. \\
			\cite{s11Rauscher2014}	&	Use ``portholes'' to view data, \ie{} aggregate data for the user to view but do no disclose the dataset. \\
			\cite{s11Rauscher2014}	&	Declassify data when output is requested by the user, hereby removing identifiable data. \\
			\cite{s16Ma2013}	&	Separate identifying data from the dataset, used in Ma \cite{s16Ma2013}  to optimise search algorithms. Non-identifying data is available for fast search, after making a selection identifying data is appended before outputting. \\
			\cite{s18Kum2014}	&	Hashes can be used together with (for example) Bloom filters to provide a solution to using identifiable data in data linkage. \\
			\ref{security-interviews} 	&	Take note of standard security measures of the present and implement those. \\
	\end{longtabu}
	\par \bigskip
	This table describes the procedural and technical problems and solutions that were found. 
	The list is sorted according to type of point (either procedural or technical and either a problem or a solution).
	*: Reference, either refers to a citation (with brackets []) or an interview (section numbering \eg{} C.1).
\end{center}

%PROCEDURAL
%solution
%TODO: 6 Limited dataset under 45 CFR §164.514(e): 'Under certain circumstances, a covered entity may use and disclose protected health information (PHI) in a limited dataset for research, public health, and health care operations purposes. The privacy regulation identifies a list of identifiers that must be removed from data in order for it to be considered a “limited dataset”. Once removed, the information is not deidentified – it is still PHI governed by the privacy regulation. A data use agreement must be signed by those wishing to use limited datasets.'
	
	\chapter{Provenance Review}
	\label{provenance-review-appendix}
	
	% Styling commands for the used terms
\newcommand{\agent}{{\tt agent}}
\newcommand{\entity}{{\tt entity}}
\newcommand{\activity}{{\tt activity}}
\newcommand{\relation}{{\tt relation}}
\newcommand{\relations}{{\tt relations}}
\newcommand{\attributes}{{\tt attributes}}

\section{Data Provenance}
\label{datamodel-provenance}

A topic with growing interest in the \escience{} field is provenance, sometimes also referred to as lineage or pedigree.
It has been borrowed from the world of art where it describes the `life' of an artwork.
Mostly this will be the record of ownership but it can also describe things like restorations.
From provenance data the quality, state, and originality of the work can be discerned.
In \escience{} the same can be applied on a piece of data \cite{dsp4moreau}.

Concerning data, provenance is stored metadata describing the process by which the data got to a certain state from a specific source \cite{dsp4moreau,dsp2buneman}.
To describe the path data has taken the W3C (World Wide Web Consortium\footnote{http://www.w3.org/}) has made a standard described in the PROV Model Primer \cite{dsp8gil}.
The gist of provenance is that it is built from a small set of assertions made by the different services that are involved in  the data process \cite{dsp4moreau}.
To actually \emph{use} provenance data a standard for schemas is described which are usable for human consumption; an example is shown in figure \ref{fig:provenance-large-schema}.

Provenance is explored because it can be considered as an extension to data security.
Since provenance directly contains data traceability information, and is can be consumable by humans, it provides another level of security compared to the described solutions in the review in appendix \ref{security-review-appendix}.

\begin{figure}
	\centering
	\includegraphics[width=0.5\linewidth]{images/provenance-overview}
	\caption{
		Example model showing the three core data types (\agent{}, \entity{}, and \activity{}) and a few of all possible \relations{} between them. 
		Taken from PROV Model Primer \cite{dsp8gil}.}
	\label{fig:provenance-overview}
\end{figure}

\paragraph{The `why?' of provenance}
\label{provenance-why}

When collecting provenance  metadata has to be captured at different steps in the data process.
This creates a overhead when using a system for a specific task.
Capturing and keeping provenance data is almost never the main functionality of a system.
However, exposing the provenance data may help users (\ie{} researchers).

In \escience{} data is gathered and generated at a fast pace.
Provenance of this data can help researchers determine whether data is:

\begin{itemize}
	\item Usable in a certain context, the metadata stored can describe the different uses of a specific data item \cite{dsp1simmhan}, \eg{} types of software that accept a data item as input.
	\item Acceptable, a researcher can discern from provenance whether to trust the accuracy and timeliness and accept if for further use \cite{dsp1simmhan,dsp3buneman}.
	\item Protected by intellectual property (IP) or should be credited, as for the acceptability of data the path can also be backtracked to the original creators and/or IP holders \cite{dsp1simmhan}.
\end{itemize}

\paragraph{The `how?' of provenance}
\label{provenance-how}

The provenance building blocks (as described by the W3C \cite{dsp8gil}) consist of three core data types (\agent{}, \entity{}, and \activity{}) and \relations{} between them.
Furthermore, \attributes{} can be assigned to provide metadata for data types or \relations{}.
Figure \ref{fig:provenance-overview} shows the standardised display methods for the data types and \relations{}, \attributes{} are displayed with a `document' symbol as shown in figure \ref{fig:provenance-large-schema}.
Also, an applied provenance example is given in figure \ref{fig:provenance-large-schema}.

The concepts as described in PROV Model Primer \cite{dsp8gil} are:

\begin{itemize}
	\item Entity, physical, digital, conceptual, or another type of `thing'.
	\item Activity, the process of instantiating or the process of changing an entity.
	\item Agent, holds (a part of) the responsibility for activities and entities.
	\item Relation, describes the interaction between two instances of the data types.
\end{itemize}

\begin{figure}
	\centering
	\includegraphics[width=1.0\linewidth]{images/provenance-large-schema}
	\caption{
		Real life example model which implements the model as shown in figure \ref{fig:provenance-overview}.
		This example describes the creation of a chart, the original data used, the intermediate data generated during the process, the used software, who was responsible for the work, and who this person was working for.
		An addition to figure \ref{fig:provenance-overview} is the use of \attributes{}, these are displayed with document icons and provide metadata on the object they are related to.
		In this case that is the name and email address for one the agents and the company name for the other agent.
		Taken from PROV Model Primer \cite{dsp8gil}.
	}
	\label{fig:provenance-large-schema}
\end{figure}

\paragraph{The use of provenance for security}
\label{provenance-use}

Three types of assertions can be described which account to provenance: relationship (object B was retrieved by applying function X to object A), interaction (received object A, sent object B), and service state (it took three seconds to send object B after receiving object A) \cite{dsp4moreau}.

Different applications of provenance can be accomplished with these assertions \eg{} data quality, audit trail, replication recipes, attribution, and informational \cite{dsp1simmhan}.
Data quality uses the provenance metadata as a check, like described with the art example at the beginning of this review, the user tries to discover who did what with the data.
The same applies for an audit trail, but it is used for other ends, being able to maintain responsibilities.
Replication recipes are an extension of data quality, a user can execute the exact same steps on the same or a different piece of data to check the work or to improve on the work done.
In the case of data ownership, attribution can be used to trace who should be contacted for consent on data usage.
Lastly, informational provenance helps with data discovery and providing context for data, which helps with (for example) reuse.

There are many applications of provenance in security, different levels can be supplied by mixing computerised surveillance with human insight.
One of the clearest examples is the data audit functionality.
The necessary metadata for an audit is collected automatically during the operation of the system.
Outcomes of this audit can be partly analysed by a computer, but can also additionally be translated into an human understandable format.
Further actions that lead to actual security should be captured in standardised processes, therefore provenance is only a tool and not an security end-point.
	
	\chapter{Security Checklists}
	\label{security-appendix}
	
	\section{Checklist A}
\label{security-checklists-a}

This section gives two checklists found in the literature.
These lists give a number of points which a system should comply to in order to be secure in the sense of privacy.
The first list describes items used to test safety of the implementation of a patient-centred eHealth solution in Dehling \cite{s17Dehling2014}:
\begin{itemize}
	\item No unauthorized person must be able to access patients' information;
	\item The real identity of patients must not be revealed;
	\item Access must be limited to necessary information and data segregation must be ensured;
	\item Unnecessary access rights must be revoked;
	\item It cannot be possible to force patients to reveal information they do not want to reveal;
	\item Eavesdropping has to be prevented during transmission and storage;
	\item It must not be possible to reveal relationships between items through observation;
	\item It must be ensured that information content is as intended and not unintentionally changed;
	\item Up-to-date information must be available whenever needed;
	\item Redundancy must be employed to ensure that data can be restored;
	\item It must be possible to store information as long as it is required (even a lifetime or longer);
	\item It must be possible to restore lost information to a specific point in time;
	\item Failure of single nodes must not impede the performance of the whole service;
	\item Systems have to be adaptable to changing performance needs;
	\item There cannot be a significant delay between data entry and dissemination to patients;
	\item Accesses to and uses of information must be attributed to the respective party and it must not be possible to deny such actions afterwards;
	\item Relevant activity (e.g. document accesses) must be logged;
	\item It must be determined who is using the software and verified that they are who they claim to be;
	\item The boundaries of trusted access to the information system must be known and controlled;
	\item Unintended actions and/or activity must be detected;
	\item Unauthorized access must be avoided and access rights must be managed;
	\item Impairment of hardware (theft, natural disasters, ...) has to be prevented;
	\item System vulnerabilities must be detected;
	\item Important information has to be easily accessible;
	\item Patients have to be able to control who can access what information;
	\item Authorization details must be substitutable (loss, technological obsolescence);
	\item User ethics, obligations, and proficiency must be reinforced;
	\item In case of emergency, medical professionals must be able to access required information;
	\item Patients have to agree to uses of their information and patient consent must be managed;
	\item Patients have to be able to retrieve information stored on them.
\end{itemize}

\section{Checklist B}
\label{security-checklists-b}

The second list describes questions used to test EHR systems on security and privacy in Fernández-Alemán \cite{s8FernandezAleman2013}:
\begin{itemize}
	\item What standards and regulations does the system satisfy?
	\item Does the system use pseudo anonymity techniques?
	\item Is the user data encrypted? 
	\item What authentication systems are used? 
	\item Can access policies be overridden in the case of an emergency? 
	\item If the system needs user roles, who defines them? 
	\item Who grants the access to the data? 
	\item What kind of information is exchanged? 
	\item Are there audit logs?
	\item Are the systems' users trained in security and privacy issues? 
\end{itemize}
	
	\chapter{Identified Functions}
	\label{identified-functions}
	
	\begin{figure}[hb]
	\centering
	\includegraphics[width=1.0\linewidth]{images/functions-in-groups}
	\caption{
		All identified functions grouped by the four function groups.
		Underlying requirement analysis described in \ref{brainstorm}.
	}
	\label{fig:all-functions-grouped}
\end{figure}
	
	\chapter{Unedited Rosemary Data Model}
	\label{unedited-datamodel}
	
	\begin{figure}[ht]
	\centering
	\includegraphics[width=0.54\linewidth]{images/datamodel-unedited}
	\caption{
		The full unedited data model as implemented in the Rosemary project \protect\url{https://github.com/AMCeScience/Rosemary}.
		Describes the workspace, tagging, application, submission, datums, and notification models.
		Taken from \protect\url{https://github.com/AMCeScience/Rosemary/blob/master/docs/general/rosemary-dm.png}.
	}
	\label{fig:unedited-rosemary-dm}
\end{figure}

\end{document}
